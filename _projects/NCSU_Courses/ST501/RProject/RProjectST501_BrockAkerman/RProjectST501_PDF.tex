% Options for packages loaded elsewhere
\PassOptionsToPackage{unicode}{hyperref}
\PassOptionsToPackage{hyphens}{url}
%
\documentclass[
]{article}
\usepackage{lmodern}
\usepackage{amssymb,amsmath}
\usepackage{ifxetex,ifluatex}
\ifnum 0\ifxetex 1\fi\ifluatex 1\fi=0 % if pdftex
  \usepackage[T1]{fontenc}
  \usepackage[utf8]{inputenc}
  \usepackage{textcomp} % provide euro and other symbols
\else % if luatex or xetex
  \usepackage{unicode-math}
  \defaultfontfeatures{Scale=MatchLowercase}
  \defaultfontfeatures[\rmfamily]{Ligatures=TeX,Scale=1}
\fi
% Use upquote if available, for straight quotes in verbatim environments
\IfFileExists{upquote.sty}{\usepackage{upquote}}{}
\IfFileExists{microtype.sty}{% use microtype if available
  \usepackage[]{microtype}
  \UseMicrotypeSet[protrusion]{basicmath} % disable protrusion for tt fonts
}{}
\makeatletter
\@ifundefined{KOMAClassName}{% if non-KOMA class
  \IfFileExists{parskip.sty}{%
    \usepackage{parskip}
  }{% else
    \setlength{\parindent}{0pt}
    \setlength{\parskip}{6pt plus 2pt minus 1pt}}
}{% if KOMA class
  \KOMAoptions{parskip=half}}
\makeatother
\usepackage{xcolor}
\IfFileExists{xurl.sty}{\usepackage{xurl}}{} % add URL line breaks if available
\IfFileExists{bookmark.sty}{\usepackage{bookmark}}{\usepackage{hyperref}}
\hypersetup{
  pdftitle={R-Project\_501},
  pdfauthor={Brock Akerman},
  hidelinks,
  pdfcreator={LaTeX via pandoc}}
\urlstyle{same} % disable monospaced font for URLs
\usepackage[margin=1in]{geometry}
\usepackage{color}
\usepackage{fancyvrb}
\newcommand{\VerbBar}{|}
\newcommand{\VERB}{\Verb[commandchars=\\\{\}]}
\DefineVerbatimEnvironment{Highlighting}{Verbatim}{commandchars=\\\{\}}
% Add ',fontsize=\small' for more characters per line
\usepackage{framed}
\definecolor{shadecolor}{RGB}{248,248,248}
\newenvironment{Shaded}{\begin{snugshade}}{\end{snugshade}}
\newcommand{\AlertTok}[1]{\textcolor[rgb]{0.94,0.16,0.16}{#1}}
\newcommand{\AnnotationTok}[1]{\textcolor[rgb]{0.56,0.35,0.01}{\textbf{\textit{#1}}}}
\newcommand{\AttributeTok}[1]{\textcolor[rgb]{0.77,0.63,0.00}{#1}}
\newcommand{\BaseNTok}[1]{\textcolor[rgb]{0.00,0.00,0.81}{#1}}
\newcommand{\BuiltInTok}[1]{#1}
\newcommand{\CharTok}[1]{\textcolor[rgb]{0.31,0.60,0.02}{#1}}
\newcommand{\CommentTok}[1]{\textcolor[rgb]{0.56,0.35,0.01}{\textit{#1}}}
\newcommand{\CommentVarTok}[1]{\textcolor[rgb]{0.56,0.35,0.01}{\textbf{\textit{#1}}}}
\newcommand{\ConstantTok}[1]{\textcolor[rgb]{0.00,0.00,0.00}{#1}}
\newcommand{\ControlFlowTok}[1]{\textcolor[rgb]{0.13,0.29,0.53}{\textbf{#1}}}
\newcommand{\DataTypeTok}[1]{\textcolor[rgb]{0.13,0.29,0.53}{#1}}
\newcommand{\DecValTok}[1]{\textcolor[rgb]{0.00,0.00,0.81}{#1}}
\newcommand{\DocumentationTok}[1]{\textcolor[rgb]{0.56,0.35,0.01}{\textbf{\textit{#1}}}}
\newcommand{\ErrorTok}[1]{\textcolor[rgb]{0.64,0.00,0.00}{\textbf{#1}}}
\newcommand{\ExtensionTok}[1]{#1}
\newcommand{\FloatTok}[1]{\textcolor[rgb]{0.00,0.00,0.81}{#1}}
\newcommand{\FunctionTok}[1]{\textcolor[rgb]{0.00,0.00,0.00}{#1}}
\newcommand{\ImportTok}[1]{#1}
\newcommand{\InformationTok}[1]{\textcolor[rgb]{0.56,0.35,0.01}{\textbf{\textit{#1}}}}
\newcommand{\KeywordTok}[1]{\textcolor[rgb]{0.13,0.29,0.53}{\textbf{#1}}}
\newcommand{\NormalTok}[1]{#1}
\newcommand{\OperatorTok}[1]{\textcolor[rgb]{0.81,0.36,0.00}{\textbf{#1}}}
\newcommand{\OtherTok}[1]{\textcolor[rgb]{0.56,0.35,0.01}{#1}}
\newcommand{\PreprocessorTok}[1]{\textcolor[rgb]{0.56,0.35,0.01}{\textit{#1}}}
\newcommand{\RegionMarkerTok}[1]{#1}
\newcommand{\SpecialCharTok}[1]{\textcolor[rgb]{0.00,0.00,0.00}{#1}}
\newcommand{\SpecialStringTok}[1]{\textcolor[rgb]{0.31,0.60,0.02}{#1}}
\newcommand{\StringTok}[1]{\textcolor[rgb]{0.31,0.60,0.02}{#1}}
\newcommand{\VariableTok}[1]{\textcolor[rgb]{0.00,0.00,0.00}{#1}}
\newcommand{\VerbatimStringTok}[1]{\textcolor[rgb]{0.31,0.60,0.02}{#1}}
\newcommand{\WarningTok}[1]{\textcolor[rgb]{0.56,0.35,0.01}{\textbf{\textit{#1}}}}
\usepackage{graphicx,grffile}
\makeatletter
\def\maxwidth{\ifdim\Gin@nat@width>\linewidth\linewidth\else\Gin@nat@width\fi}
\def\maxheight{\ifdim\Gin@nat@height>\textheight\textheight\else\Gin@nat@height\fi}
\makeatother
% Scale images if necessary, so that they will not overflow the page
% margins by default, and it is still possible to overwrite the defaults
% using explicit options in \includegraphics[width, height, ...]{}
\setkeys{Gin}{width=\maxwidth,height=\maxheight,keepaspectratio}
% Set default figure placement to htbp
\makeatletter
\def\fps@figure{htbp}
\makeatother
\setlength{\emergencystretch}{3em} % prevent overfull lines
\providecommand{\tightlist}{%
  \setlength{\itemsep}{0pt}\setlength{\parskip}{0pt}}
\setcounter{secnumdepth}{-\maxdimen} % remove section numbering

\title{R-Project\_501}
\author{Brock Akerman}
\date{11/16/2020}

\begin{document}
\maketitle

\textbf{Part 1: ``Visualizing Convergence in Probability''}

Show that the minimum order statistic converges in probability to 0.
Hint: We know the CDF of an exponential and how to find the CDF of the
minimum order statistic Y(1). Start with the probability you want to
show from the definition of convergence in probability to 0, i.e.,
P(\textbar Y(1) − 0\textbar{} \textless{} ε) and take the limit as n
goes to infinity and show that this probability converges to 1.

To visualize this we'll simulate data and approximate the probability
statement proven in the previous part. -- For a sample size of n = 1,
generate N = 1000 data sets from an exp(1) distribution -- For each data
set, find the minimum value (for a sample of size 1 that will just be
the value itself) -- Save these minimum values for plotting

\begin{Shaded}
\begin{Highlighting}[]
\KeywordTok{set.seed}\NormalTok{(}\DecValTok{841}\NormalTok{)}
\NormalTok{n <-}\StringTok{ }\DecValTok{1}
\NormalTok{N <-}\StringTok{ }\DecValTok{1000}
\NormalTok{minmat <-}\StringTok{ }\KeywordTok{matrix}\NormalTok{(}\DataTypeTok{ncol=}\DecValTok{1}\NormalTok{,}\DataTypeTok{nrow=}\DecValTok{1000}\NormalTok{)}
\ControlFlowTok{for}\NormalTok{ (i }\ControlFlowTok{in} \DecValTok{1}\OperatorTok{:}\NormalTok{N)\{}
\NormalTok{    ds <-}\StringTok{ }\KeywordTok{replicate}\NormalTok{(i,(}\KeywordTok{rexp}\NormalTok{(n,}\DataTypeTok{rate =} \DecValTok{1}\NormalTok{)))}
\NormalTok{    minmat[i] <-}\StringTok{ }\KeywordTok{min}\NormalTok{(ds)}
\NormalTok{    \}}
\NormalTok{i <-}\StringTok{ }\DecValTok{1}
\end{Highlighting}
\end{Shaded}

Now set ε = 0.05. Next approximate the probability of interest
P(\textbar Y(1) − 0\textbar{} \textless= ε) using the N = 1000 simulated
minimum values. (This is a Monte Carlo estimate of the probability.)
Save this probability

\begin{Shaded}
\begin{Highlighting}[]
\NormalTok{ε <-}\StringTok{ }\FloatTok{0.05}
\NormalTok{TF_values <-}\StringTok{ }\NormalTok{(}\KeywordTok{abs}\NormalTok{(minmat}\DecValTok{-0}\NormalTok{) }\OperatorTok{<=}\StringTok{ }\NormalTok{ε) }\CommentTok{#Assign Boolean values to a vector}
\NormalTok{dsprob <-}\StringTok{ }\KeywordTok{sum}\NormalTok{(TF_values)}\OperatorTok{/}\NormalTok{N }\CommentTok{#Probability of true values from our 1000 dataset minimums. }
\end{Highlighting}
\end{Shaded}

Repeat the above simulation and approximation of the probability of
interest for n = 2, 3, \ldots, 50.

\begin{Shaded}
\begin{Highlighting}[]
\NormalTok{fullminmat <-}\StringTok{ }\KeywordTok{matrix}\NormalTok{(}\DataTypeTok{ncol=}\DecValTok{50}\NormalTok{,}\DataTypeTok{nrow=}\DecValTok{1000}\NormalTok{)}
\ControlFlowTok{for}\NormalTok{ (j }\ControlFlowTok{in} \DecValTok{1}\OperatorTok{:}\DecValTok{50}\NormalTok{)\{}
  \ControlFlowTok{for}\NormalTok{ (k }\ControlFlowTok{in} \DecValTok{1}\OperatorTok{:}\NormalTok{N) \{}
\NormalTok{  fullds <-}\StringTok{ }\KeywordTok{replicate}\NormalTok{(k,(}\KeywordTok{rexp}\NormalTok{(j,}\DataTypeTok{rate =} \DecValTok{1}\NormalTok{)))}
\NormalTok{  fullminmat[k,j] <-}\StringTok{ }\KeywordTok{min}\NormalTok{(fullds) \} }\CommentTok{#Creates our }
\NormalTok{\}}

\NormalTok{TFmat <-}\StringTok{ }\KeywordTok{abs}\NormalTok{(fullminmat}\DecValTok{-0}\NormalTok{)}\OperatorTok{<=}\StringTok{ }\NormalTok{ε}
\NormalTok{fulldsprob <-}\StringTok{ }\KeywordTok{matrix}\NormalTok{(}\KeywordTok{colSums}\NormalTok{(TFmat, }\DataTypeTok{na.rm=}\OtherTok{FALSE}\NormalTok{, }\DataTypeTok{dims=}\DecValTok{1}\NormalTok{),}\DataTypeTok{ncol=}\DecValTok{50}\NormalTok{,}\DataTypeTok{nrow=}\DecValTok{1}\NormalTok{)}
\NormalTok{probmatrix <-}\StringTok{ }\NormalTok{fulldsprob}\OperatorTok{/}\NormalTok{N}
\end{Highlighting}
\end{Shaded}

Now create a plot with the sample size on the x-axis and the probability
of interest on the y-axis. The plot should have an appropriate title and
appropriate axis labels. In a comment explain how this plot can help
someone understand convergence in probability to a constant.

\begin{Shaded}
\begin{Highlighting}[]
\KeywordTok{plot}\NormalTok{(}\DataTypeTok{x =} \DecValTok{1}\OperatorTok{:}\DecValTok{50}\NormalTok{,}
     \DataTypeTok{y=}\NormalTok{probmatrix,}
     \DataTypeTok{main=}\StringTok{"Monte Carlo Plot Simulation of Convergence"}\NormalTok{, }
     \DataTypeTok{xlab=}\StringTok{"Sample Size"}\NormalTok{, }
     \DataTypeTok{ylab=}\StringTok{"probability of Interest"}\NormalTok{, }
\NormalTok{     )}
\end{Highlighting}
\end{Shaded}

\includegraphics{RProjectST501_PDF_files/figure-latex/unnamed-chunk-5-1.pdf}

Now for each value of n\{1, 5, 10, 25, 50\}, draw one histogram of the
minimum values for a sample of size n.~You will thus have 5 histogram
plots and, for example, the histogram plot for n = 10 will be a
histogram plot of the N = 1000 minimum values for the N = 1000 samples
of size n = 10. In a comment, explain how these histogram plots (for n
changing), can help someone understand convergence in probability to a
constant.

\begin{Shaded}
\begin{Highlighting}[]
\NormalTok{F01 <-}\StringTok{ }\KeywordTok{sort}\NormalTok{(}\KeywordTok{replicate}\NormalTok{(N, }\KeywordTok{min}\NormalTok{(}\KeywordTok{rexp}\NormalTok{(}\DecValTok{1}\NormalTok{,}\DataTypeTok{rate =} \DecValTok{1}\NormalTok{))))}
\NormalTok{FM01 <-}\StringTok{ }\KeywordTok{matrix}\NormalTok{(F01,}\DataTypeTok{nrow=}\NormalTok{N,}\DataTypeTok{ncol=}\DecValTok{1}\NormalTok{)}
\NormalTok{probability01 <-}\StringTok{ }\KeywordTok{sum}\NormalTok{(FM01}\OperatorTok{<=}\NormalTok{ε)}\OperatorTok{/}\NormalTok{N}
\KeywordTok{hist}\NormalTok{(FM01, }\DataTypeTok{breaks=}\DecValTok{20}\NormalTok{, }\DataTypeTok{prob=}\OtherTok{FALSE}\NormalTok{, }\DataTypeTok{xlim=}\KeywordTok{c}\NormalTok{(}\DecValTok{0}\NormalTok{,}\DecValTok{6}\NormalTok{), }\DataTypeTok{ylim=}\KeywordTok{c}\NormalTok{(}\DecValTok{0}\NormalTok{,}\DecValTok{400}\NormalTok{), }\DataTypeTok{main=}\StringTok{"One-Thousand minimum values of an exp(1) distr. with size n=1"}\NormalTok{, }\DataTypeTok{xlab=}\StringTok{"Minimum Values"}\NormalTok{, }\DataTypeTok{ylab=}\StringTok{"Frequency"}\NormalTok{)}
\end{Highlighting}
\end{Shaded}

\includegraphics{RProjectST501_PDF_files/figure-latex/unnamed-chunk-6-1.pdf}

\begin{Shaded}
\begin{Highlighting}[]
\NormalTok{F05 <-}\StringTok{ }\KeywordTok{sort}\NormalTok{(}\KeywordTok{replicate}\NormalTok{(N, }\KeywordTok{min}\NormalTok{(}\KeywordTok{rexp}\NormalTok{(}\DecValTok{5}\NormalTok{,}\DataTypeTok{rate =} \DecValTok{1}\NormalTok{))))}
\NormalTok{FM05 <-}\StringTok{ }\KeywordTok{matrix}\NormalTok{(F05,}\DataTypeTok{nrow=}\NormalTok{N,}\DataTypeTok{ncol=}\DecValTok{1}\NormalTok{)}
\NormalTok{probability05 <-}\StringTok{ }\KeywordTok{sum}\NormalTok{(FM05}\OperatorTok{<=}\NormalTok{ε)}\OperatorTok{/}\NormalTok{N}
\KeywordTok{hist}\NormalTok{(FM05, }\DataTypeTok{breaks=}\DecValTok{20}\NormalTok{, }\DataTypeTok{prob=}\OtherTok{FALSE}\NormalTok{, }\DataTypeTok{xlim=}\KeywordTok{c}\NormalTok{(}\DecValTok{0}\NormalTok{,}\DecValTok{6}\NormalTok{), }\DataTypeTok{ylim=}\KeywordTok{c}\NormalTok{(}\DecValTok{0}\NormalTok{,}\DecValTok{400}\NormalTok{), }\DataTypeTok{main=}\StringTok{"One-Thousand minimum values of an exp(1) distr. with size n=5"}\NormalTok{, }\DataTypeTok{xlab=}\StringTok{"Minimum Values"}\NormalTok{, }\DataTypeTok{ylab=}\StringTok{"Frequency"}\NormalTok{)}
\end{Highlighting}
\end{Shaded}

\includegraphics{RProjectST501_PDF_files/figure-latex/unnamed-chunk-7-1.pdf}

\begin{Shaded}
\begin{Highlighting}[]
\NormalTok{F10 <-}\StringTok{ }\KeywordTok{sort}\NormalTok{(}\KeywordTok{replicate}\NormalTok{(N, }\KeywordTok{min}\NormalTok{(}\KeywordTok{rexp}\NormalTok{(}\DecValTok{10}\NormalTok{,}\DataTypeTok{rate =} \DecValTok{1}\NormalTok{))))}
\NormalTok{FM10 <-}\StringTok{ }\KeywordTok{matrix}\NormalTok{(F10,}\DataTypeTok{nrow=}\NormalTok{N,}\DataTypeTok{ncol=}\DecValTok{1}\NormalTok{)}
\NormalTok{probability10 <-}\StringTok{ }\KeywordTok{sum}\NormalTok{(FM10}\OperatorTok{<=}\NormalTok{ε)}\OperatorTok{/}\NormalTok{N}
\KeywordTok{hist}\NormalTok{(FM10, }\DataTypeTok{breaks=}\DecValTok{20}\NormalTok{, }\DataTypeTok{prob=}\OtherTok{FALSE}\NormalTok{, }\DataTypeTok{xlim=}\KeywordTok{c}\NormalTok{(}\DecValTok{0}\NormalTok{,}\DecValTok{6}\NormalTok{), }\DataTypeTok{ylim=}\KeywordTok{c}\NormalTok{(}\DecValTok{0}\NormalTok{,}\DecValTok{400}\NormalTok{), }\DataTypeTok{main=}\StringTok{"One-Thousand minimum values of an exp(1) distr. with size n=10"}\NormalTok{, }\DataTypeTok{xlab=}\StringTok{"Minimum Values"}\NormalTok{, }\DataTypeTok{ylab=}\StringTok{"Frequency"}\NormalTok{)}
\end{Highlighting}
\end{Shaded}

\includegraphics{RProjectST501_PDF_files/figure-latex/unnamed-chunk-8-1.pdf}

\begin{Shaded}
\begin{Highlighting}[]
\NormalTok{F25 <-}\StringTok{ }\KeywordTok{sort}\NormalTok{(}\KeywordTok{replicate}\NormalTok{(N, }\KeywordTok{min}\NormalTok{(}\KeywordTok{rexp}\NormalTok{(}\DecValTok{25}\NormalTok{,}\DataTypeTok{rate =} \DecValTok{1}\NormalTok{))))}
\NormalTok{FM25 <-}\StringTok{ }\KeywordTok{matrix}\NormalTok{(F25,}\DataTypeTok{nrow=}\NormalTok{N,}\DataTypeTok{ncol=}\DecValTok{1}\NormalTok{)}
\NormalTok{probability25 <-}\StringTok{ }\KeywordTok{sum}\NormalTok{(FM25}\OperatorTok{<=}\NormalTok{ε)}\OperatorTok{/}\NormalTok{N}
\KeywordTok{hist}\NormalTok{(FM25, }\DataTypeTok{breaks=}\DecValTok{20}\NormalTok{, }\DataTypeTok{prob=}\OtherTok{FALSE}\NormalTok{, }\DataTypeTok{xlim=}\KeywordTok{c}\NormalTok{(}\DecValTok{0}\NormalTok{,}\DecValTok{6}\NormalTok{), }\DataTypeTok{ylim=}\KeywordTok{c}\NormalTok{(}\DecValTok{0}\NormalTok{,}\DecValTok{400}\NormalTok{), }\DataTypeTok{main=}\StringTok{"One-Thousand minimum values of an exp(1) distr. with size n=25"}\NormalTok{, }\DataTypeTok{xlab=}\StringTok{"Minimum Values"}\NormalTok{, }\DataTypeTok{ylab=}\StringTok{"Frequency"}\NormalTok{)}
\end{Highlighting}
\end{Shaded}

\includegraphics{RProjectST501_PDF_files/figure-latex/unnamed-chunk-9-1.pdf}

\begin{Shaded}
\begin{Highlighting}[]
\NormalTok{F50 <-}\StringTok{ }\KeywordTok{sort}\NormalTok{(}\KeywordTok{replicate}\NormalTok{(N, }\KeywordTok{min}\NormalTok{(}\KeywordTok{rexp}\NormalTok{(}\DecValTok{50}\NormalTok{,}\DataTypeTok{rate =} \DecValTok{1}\NormalTok{))))}
\NormalTok{FM50 <-}\StringTok{ }\KeywordTok{matrix}\NormalTok{(F50,}\DataTypeTok{nrow=}\NormalTok{N,}\DataTypeTok{ncol=}\DecValTok{1000}\NormalTok{)}
\NormalTok{probability50 <-}\StringTok{ }\KeywordTok{sum}\NormalTok{(FM50}\OperatorTok{<=}\NormalTok{ε)}\OperatorTok{/}\NormalTok{N}
\KeywordTok{hist}\NormalTok{(FM50, }\DataTypeTok{breaks=}\DecValTok{20}\NormalTok{, }\DataTypeTok{prob=}\OtherTok{FALSE}\NormalTok{, }\DataTypeTok{xlim=}\KeywordTok{c}\NormalTok{(}\DecValTok{0}\NormalTok{,}\DecValTok{10}\NormalTok{), }\DataTypeTok{ylim=}\KeywordTok{c}\NormalTok{(}\DecValTok{0}\NormalTok{,}\DecValTok{1000}\NormalTok{), }\DataTypeTok{main=}\StringTok{"One-Thousand minimum values of an exp(1) distr. with size n=50"}\NormalTok{, }\DataTypeTok{xlab=}\StringTok{"Minimum Values"}\NormalTok{, }\DataTypeTok{ylab=}\StringTok{"Frequency"}\NormalTok{)}
\end{Highlighting}
\end{Shaded}

\includegraphics{RProjectST501_PDF_files/figure-latex/unnamed-chunk-10-1.pdf}

\textbf{Discussion Question:} With a small sampling of data, the Exp(1)
shows some variability which manifests itself as a right-tail in the
distribution. However, as sample size increases the Exp(1) distirbution
begins to converge on zero. With enough samples, such is the case when n
= 50, the distribution nearly resembles a vertical line about zero.
There are values within the minimum value dataset that are greater than
zero. There are simply too many values at or nearer to zero that the
exp(1) looks almost tail-less. In fact, that is precisely what is
occuring. In a convergence in probability to a constant--in the case of
Exp(1) is zero--the Exp(1) is converging to a number while its variance
collapses to zero. When looking at our ordered datasets, from n=1 to
n=50, the distribution is consistently becoming thinner and
``squishing'' toward the y-axis.

\textbf{Part 2: ``Visualizing Convergence in Distribution''} This part
will consider how well the Central Limit Theorem applies to sample means
from Poisson data

Consider a sample size of n = 5 from a Poisson distribution with rate
parameter \(\lambda\) = 1. Generate N = 50000 data sets of size n from
the Poisson distribution.

\begin{Shaded}
\begin{Highlighting}[]
\NormalTok{m =}\StringTok{ }\DecValTok{5}
\NormalTok{lambda =}\StringTok{ }\DecValTok{1}
\NormalTok{M =}\StringTok{ }\DecValTok{50000}
\NormalTok{ds2 <-}\StringTok{ }\KeywordTok{matrix}\NormalTok{(}\KeywordTok{replicate}\NormalTok{(M,(}\KeywordTok{rpois}\NormalTok{(m,}\DecValTok{1}\NormalTok{))),}\DataTypeTok{nrow=}\DecValTok{50000}\NormalTok{,}\DataTypeTok{ncol=}\DecValTok{5}\NormalTok{)}
\end{Highlighting}
\end{Shaded}

For each data set, find the sample mean value. Hint If you saved the
above data in a large matrix the colMeans or rowMeans functions can be
handy here), e.g., n \textless- 5 N \textless- 50000 A matrix of N =
50000 rows, each rows containing n = 5 sample from a Poisson
distribution with rate 1 X \textless- matrix(rpois(n*N, \(\lambda\) =
1), nrow = N), rowMeans(X) - This will give you a vector of N = 50000
sample means.

\begin{Shaded}
\begin{Highlighting}[]
\NormalTok{rMeans <-}\StringTok{ }\KeywordTok{matrix}\NormalTok{(}\KeywordTok{rowMeans}\NormalTok{(ds2),}\DataTypeTok{nrow=}\DecValTok{50000}\NormalTok{,}\DataTypeTok{ncol=}\DecValTok{1}\NormalTok{);}
\end{Highlighting}
\end{Shaded}

Create a histogram of the sample means. Make the bins of appropriate
width so that each bin only has one value of the support. For instance,
the possible values for the sample mean here are 0, 0.2, 0.4, 0.6, . .
.. Make sure that each bar only has one of these values included (so the
bins would go from say −0.1 to 0.1, 0.1 to 0.3, 0.3 to 0.5. The use of
the breaks argument for the histogram function hist will be helpful
here. The central limit theorem says that \(\bar{X}\) \textasciitilde{}
N(\(\lambda\), \(\lambda\)/n) when \(\bar{X}\) is the sample mean of n
i.i.d Pois(\(\lambda\)) random variables. Overlay this large-sample
distribution on the histogram (Hint:use freq = FALSE in your histogram
and the curve function with add = TRUE to overlay the normal
distribution). All plots should have appropriate titles and axis labels.

\begin{Shaded}
\begin{Highlighting}[]
\KeywordTok{par}\NormalTok{(}\DataTypeTok{bg=}\StringTok{"gray"}\NormalTok{)}
\NormalTok{binlen <-}\StringTok{ }\KeywordTok{c}\NormalTok{(}\OperatorTok{-}\FloatTok{0.1}\NormalTok{,}\FloatTok{0.1}\NormalTok{,}\FloatTok{0.3}\NormalTok{,}\FloatTok{0.5}\NormalTok{,}\FloatTok{0.7}\NormalTok{,}\FloatTok{0.9}\NormalTok{,}\FloatTok{1.1}\NormalTok{,}\FloatTok{1.3}\NormalTok{,}\FloatTok{1.5}\NormalTok{,}\FloatTok{1.7}\NormalTok{,}\FloatTok{1.9}\NormalTok{,}\FloatTok{2.1}\NormalTok{,}\FloatTok{2.3}\NormalTok{,}\FloatTok{2.5}\NormalTok{,}\FloatTok{2.7}\NormalTok{,}\FloatTok{2.9}\NormalTok{,}\FloatTok{3.1}\NormalTok{,}\FloatTok{3.3}\NormalTok{,}\FloatTok{3.5}\NormalTok{,}\FloatTok{3.7}\NormalTok{,}\FloatTok{3.9}\NormalTok{,}\FloatTok{4.1}\NormalTok{);}
\KeywordTok{hist}\NormalTok{(rMeans ,}\DataTypeTok{breaks=}\NormalTok{binlen ,}\DataTypeTok{main=}\StringTok{"Visualizing Convergence in Distribution"}\NormalTok{ ,}\DataTypeTok{xlab=}\StringTok{"sample poisson means"}\NormalTok{ ,}\DataTypeTok{xlim=}\KeywordTok{c}\NormalTok{(}\OperatorTok{-}\FloatTok{0.5}\NormalTok{,}\FloatTok{2.5}\NormalTok{) ,}\DataTypeTok{ylab=}\StringTok{"density"}\NormalTok{ ,}\DataTypeTok{ylim=}\KeywordTok{c}\NormalTok{(}\DecValTok{0}\NormalTok{,}\FloatTok{1.1}\NormalTok{) ,}\DataTypeTok{freq =} \OtherTok{FALSE}\NormalTok{)}
\NormalTok{x <-}\StringTok{ }\KeywordTok{seq}\NormalTok{(}\OperatorTok{-}\DecValTok{4}\NormalTok{,}\DecValTok{5}\NormalTok{, }\DataTypeTok{by =} \FloatTok{0.1}\NormalTok{)}
\NormalTok{y <-}\StringTok{ }\KeywordTok{dnorm}\NormalTok{(x, }\DataTypeTok{mean =}\NormalTok{ lambda, }\DataTypeTok{sd =}\NormalTok{ lambda}\OperatorTok{/}\KeywordTok{sqrt}\NormalTok{(m))}
\KeywordTok{curve}\NormalTok{(}\KeywordTok{dnorm}\NormalTok{(x, }\DataTypeTok{mean=}\NormalTok{lambda, }\DataTypeTok{sd=}\NormalTok{lambda}\OperatorTok{/}\KeywordTok{sqrt}\NormalTok{(m)), }\DataTypeTok{add=}\OtherTok{TRUE}\NormalTok{, }\DataTypeTok{col=}\StringTok{"orange"}\NormalTok{, }\DataTypeTok{lwd=}\DecValTok{2}\NormalTok{)}
\end{Highlighting}
\end{Shaded}

\includegraphics{RProjectST501_PDF_files/figure-latex/unnamed-chunk-13-1.pdf}

Use the N = 50000 {[}mean{]} values to approximate the probability that
\(\bar{X}\) is greater than or equal to \(\lambda\) +
2\(\lambda\)/sqrt(n). Also report this probability as approximated by
the normal distribution.

\begin{Shaded}
\begin{Highlighting}[]
\NormalTok{TF_values}\FloatTok{.2}\NormalTok{ <-}\StringTok{ }\NormalTok{rMeans }\OperatorTok{>=}\StringTok{ }\NormalTok{(lambda}\OperatorTok{+}\NormalTok{((}\DecValTok{2}\OperatorTok{*}\NormalTok{lambda)}\OperatorTok{/}\KeywordTok{sqrt}\NormalTok{(m)))}
\NormalTok{TFprob <-}\StringTok{ }\KeywordTok{sum}\NormalTok{(TF_values}\FloatTok{.2}\NormalTok{)}\OperatorTok{/}\NormalTok{M}
\KeywordTok{print}\NormalTok{(}\DecValTok{1}\OperatorTok{-}\NormalTok{TFprob)}
\end{Highlighting}
\end{Shaded}

\begin{verbatim}
## [1] 0.967
\end{verbatim}

\begin{Shaded}
\begin{Highlighting}[]
\NormalTok{NormProb <-}\StringTok{ }\DecValTok{1}\OperatorTok{-}\KeywordTok{dnorm}\NormalTok{((lambda}\OperatorTok{+}\NormalTok{((}\DecValTok{2}\OperatorTok{*}\NormalTok{lambda)}\OperatorTok{/}\KeywordTok{sqrt}\NormalTok{(m))),}\DataTypeTok{mean=}\NormalTok{lambda,}\DataTypeTok{sd=}\NormalTok{(lambda}\OperatorTok{/}\KeywordTok{sqrt}\NormalTok{(m)))}
\KeywordTok{print}\NormalTok{(NormProb)}
\end{Highlighting}
\end{Shaded}

\begin{verbatim}
## [1] 0.8792725
\end{verbatim}

Repeat the above for n = 10, n = 30, and n = 100. FOR m = 1,\(\lambda\)
= 1

\begin{Shaded}
\begin{Highlighting}[]
\NormalTok{m1 =}\StringTok{ }\DecValTok{1}\NormalTok{; lambda}\FloatTok{.1}\NormalTok{ =}\StringTok{ }\DecValTok{1}\NormalTok{; M =}\StringTok{ }\DecValTok{50000}
\NormalTok{ds1 <-}\StringTok{ }\KeywordTok{matrix}\NormalTok{(}\KeywordTok{replicate}\NormalTok{(M,(}\KeywordTok{rpois}\NormalTok{(m1,lambda}\FloatTok{.1}\NormalTok{))),}\DataTypeTok{nrow=}\DecValTok{50000}\NormalTok{,}\DataTypeTok{ncol=}\DecValTok{5}\NormalTok{)}
\NormalTok{rMeans1 <-}\StringTok{ }\KeywordTok{matrix}\NormalTok{(}\KeywordTok{rowMeans}\NormalTok{(ds1),}\DataTypeTok{nrow=}\DecValTok{50000}\NormalTok{,}\DataTypeTok{ncol=}\DecValTok{1}\NormalTok{)}
\NormalTok{binlen <-}\StringTok{ }\KeywordTok{c}\NormalTok{(}\OperatorTok{-}\FloatTok{0.5}\NormalTok{,}\FloatTok{0.5}\NormalTok{,}\FloatTok{1.5}\NormalTok{,}\FloatTok{2.5}\NormalTok{,}\FloatTok{3.5}\NormalTok{,}\FloatTok{4.5}\NormalTok{,}\FloatTok{5.5}\NormalTok{,}\FloatTok{6.5}\NormalTok{,}\FloatTok{7.5}\NormalTok{,}\FloatTok{8.5}\NormalTok{,}\FloatTok{9.5}\NormalTok{,}\FloatTok{10.5}\NormalTok{,}\FloatTok{11.5}\NormalTok{,}\FloatTok{12.5}\NormalTok{,}\FloatTok{13.5}\NormalTok{,}\FloatTok{14.5}\NormalTok{,}\FloatTok{15.5}\NormalTok{,}\FloatTok{16.5}\NormalTok{,}\FloatTok{17.5}\NormalTok{,}\FloatTok{18.5}\NormalTok{,}\FloatTok{19.5}\NormalTok{,}\FloatTok{20.5}\NormalTok{,}\FloatTok{21.5}\NormalTok{,}\FloatTok{22.5}\NormalTok{);}
\KeywordTok{hist}\NormalTok{(rMeans1,}\DataTypeTok{breaks=}\NormalTok{binlen ,}\DataTypeTok{main=}\KeywordTok{expression}\NormalTok{(}\KeywordTok{paste}\NormalTok{(}\StringTok{"Pois("}\NormalTok{,lambda,}\StringTok{") ~ N("}\NormalTok{,lambda,}\StringTok{","}\NormalTok{,lambda,}\StringTok{"/m) curve of 50,000 means where m=1,"}\NormalTok{,lambda,}\StringTok{"=1"}\NormalTok{))  ,}\DataTypeTok{xlab=}\StringTok{"sample poisson means"}\NormalTok{ ,}\DataTypeTok{xlim=}\KeywordTok{c}\NormalTok{(}\DecValTok{0}\NormalTok{,}\DecValTok{5}\NormalTok{) ,}\DataTypeTok{ylab=}\StringTok{"density"}\NormalTok{ ,}\DataTypeTok{ylim=}\KeywordTok{c}\NormalTok{(}\DecValTok{0}\NormalTok{,}\DecValTok{1}\NormalTok{) ,}\DataTypeTok{freq =} \OtherTok{FALSE}\NormalTok{, }\DataTypeTok{col=}\KeywordTok{viridis}\NormalTok{(}\DecValTok{4}\NormalTok{))}
\NormalTok{x <-}\StringTok{ }\KeywordTok{seq}\NormalTok{(}\DecValTok{0}\NormalTok{,}\DecValTok{2}\NormalTok{, }\DataTypeTok{by =} \FloatTok{0.2}\NormalTok{)}
\NormalTok{y1 <-}\StringTok{ }\KeywordTok{dnorm}\NormalTok{(x, }\DataTypeTok{mean=}\NormalTok{lambda}\FloatTok{.1}\NormalTok{, }\DataTypeTok{sd=}\NormalTok{lambda}\FloatTok{.1}\OperatorTok{/}\NormalTok{(}\KeywordTok{sqrt}\NormalTok{(m1)))}
\KeywordTok{curve}\NormalTok{(}\KeywordTok{dnorm}\NormalTok{(x, }\DataTypeTok{mean=}\NormalTok{lambda}\FloatTok{.1}\NormalTok{, }\DataTypeTok{sd=}\NormalTok{(lambda}\FloatTok{.1}\OperatorTok{/}\KeywordTok{sqrt}\NormalTok{(m1))), }\DataTypeTok{add=}\OtherTok{TRUE}\NormalTok{, }\DataTypeTok{col=}\StringTok{"#cc0000"}\NormalTok{, }\DataTypeTok{lwd=}\DecValTok{2}\NormalTok{)}
\end{Highlighting}
\end{Shaded}

\includegraphics{RProjectST501_PDF_files/figure-latex/unnamed-chunk-15-1.pdf}

\begin{Shaded}
\begin{Highlighting}[]
\NormalTok{TF_values}\FloatTok{.1}\NormalTok{ <-}\StringTok{ }\NormalTok{rMeans1 }\OperatorTok{>=}\StringTok{ }\NormalTok{(lambda}\FloatTok{.1}\OperatorTok{+}\NormalTok{((}\DecValTok{2}\OperatorTok{*}\NormalTok{lambda}\FloatTok{.1}\NormalTok{)}\OperatorTok{/}\KeywordTok{sqrt}\NormalTok{(m1)))}
\NormalTok{TFprob1 <-}\StringTok{ }\KeywordTok{sum}\NormalTok{(TF_values}\FloatTok{.1}\NormalTok{)}\OperatorTok{/}\NormalTok{M; }\KeywordTok{print}\NormalTok{(TFprob1)}
\end{Highlighting}
\end{Shaded}

\begin{verbatim}
## [1] 0.07924
\end{verbatim}

\begin{Shaded}
\begin{Highlighting}[]
\NormalTok{NormProb1 <-}\StringTok{ }\KeywordTok{dnorm}\NormalTok{((lambda}\FloatTok{.1}\OperatorTok{+}\NormalTok{((}\DecValTok{2}\OperatorTok{*}\NormalTok{lambda}\FloatTok{.1}\NormalTok{)}\OperatorTok{/}\KeywordTok{sqrt}\NormalTok{(m1))),}\DataTypeTok{mean=}\NormalTok{lambda}\FloatTok{.1}\NormalTok{,}\DataTypeTok{sd=}\NormalTok{(lambda}\FloatTok{.1}\OperatorTok{/}\KeywordTok{sqrt}\NormalTok{(m1))); }\KeywordTok{print}\NormalTok{(NormProb1)}
\end{Highlighting}
\end{Shaded}

\begin{verbatim}
## [1] 0.05399097
\end{verbatim}

FOR m = 10,\(\lambda\) = 1

\begin{Shaded}
\begin{Highlighting}[]
\NormalTok{m10 =}\StringTok{ }\DecValTok{10}\NormalTok{; lambda}\FloatTok{.1}\NormalTok{ =}\StringTok{ }\DecValTok{1}\NormalTok{; M =}\StringTok{ }\DecValTok{50000}
\NormalTok{ds10 <-}\StringTok{ }\KeywordTok{matrix}\NormalTok{(}\KeywordTok{replicate}\NormalTok{(M,(}\KeywordTok{rpois}\NormalTok{(m10,}\DecValTok{1}\NormalTok{))),}\DataTypeTok{nrow=}\DecValTok{50000}\NormalTok{,}\DataTypeTok{ncol=}\DecValTok{5}\NormalTok{)}
\NormalTok{rMeans10 <-}\StringTok{ }\KeywordTok{matrix}\NormalTok{(}\KeywordTok{rowMeans}\NormalTok{(ds10),}\DataTypeTok{nrow=}\DecValTok{50000}\NormalTok{,}\DataTypeTok{ncol=}\DecValTok{1}\NormalTok{)}
\NormalTok{binlen <-}\StringTok{ }\KeywordTok{c}\NormalTok{(}\OperatorTok{-}\FloatTok{5.1}\NormalTok{,}\OperatorTok{-}\FloatTok{4.9}\NormalTok{,}\OperatorTok{-}\FloatTok{4.7}\NormalTok{,}\OperatorTok{-}\FloatTok{4.5}\NormalTok{,}\OperatorTok{-}\FloatTok{4.3}\NormalTok{,}\OperatorTok{-}\FloatTok{4.1}\NormalTok{,}\OperatorTok{-}\FloatTok{3.9}\NormalTok{,}\OperatorTok{-}\FloatTok{3.7}\NormalTok{,}\OperatorTok{-}\FloatTok{3.5}\NormalTok{,}\OperatorTok{-}\FloatTok{3.3}\NormalTok{,}\OperatorTok{-}\FloatTok{3.1}\NormalTok{,}\OperatorTok{-}\FloatTok{2.9}\NormalTok{,}\OperatorTok{-}\FloatTok{2.7}\NormalTok{,}\OperatorTok{-}\FloatTok{2.5}\NormalTok{,}\OperatorTok{-}\FloatTok{2.3}\NormalTok{,}\OperatorTok{-}\FloatTok{2.1}\NormalTok{,}\OperatorTok{-}\FloatTok{1.9}\NormalTok{,}\OperatorTok{-}\FloatTok{1.7}\NormalTok{,}\OperatorTok{-}\FloatTok{1.5}\NormalTok{,}\OperatorTok{-}\FloatTok{1.3}\NormalTok{,}\OperatorTok{-}\FloatTok{1.1}\NormalTok{,}\OperatorTok{-}\FloatTok{0.9}\NormalTok{,}\OperatorTok{-}\FloatTok{0.7}\NormalTok{,}\OperatorTok{-}\FloatTok{0.5}\NormalTok{,}\OperatorTok{-}\FloatTok{0.3}\NormalTok{,}\OperatorTok{-}\FloatTok{0.1}\NormalTok{,}\FloatTok{0.1}\NormalTok{,}\FloatTok{0.3}\NormalTok{,}\FloatTok{0.5}\NormalTok{,}\FloatTok{0.7}\NormalTok{,}\FloatTok{0.9}\NormalTok{,}\FloatTok{1.1}\NormalTok{,}\FloatTok{1.3}\NormalTok{,}\FloatTok{1.5}\NormalTok{,}\FloatTok{1.7}\NormalTok{,}\FloatTok{1.9}\NormalTok{,}\FloatTok{2.1}\NormalTok{,}\FloatTok{2.3}\NormalTok{,}\FloatTok{2.5}\NormalTok{,}\FloatTok{2.7}\NormalTok{,}\FloatTok{2.9}\NormalTok{,}\FloatTok{3.1}\NormalTok{,}\FloatTok{3.3}\NormalTok{,}\FloatTok{3.5}\NormalTok{,}\FloatTok{3.7}\NormalTok{,}\FloatTok{3.9}\NormalTok{,}\FloatTok{4.1}\NormalTok{,}\FloatTok{4.3}\NormalTok{,}\FloatTok{4.5}\NormalTok{,}\FloatTok{4.7}\NormalTok{,}\FloatTok{4.9}\NormalTok{,}\FloatTok{5.1}\NormalTok{)}
\KeywordTok{hist}\NormalTok{(rMeans10,}\DataTypeTok{breaks=}\NormalTok{binlen ,}\DataTypeTok{main=}\KeywordTok{expression}\NormalTok{(}\KeywordTok{paste}\NormalTok{(}\StringTok{"Pois("}\NormalTok{,lambda,}\StringTok{") ~ N("}\NormalTok{,lambda,}\StringTok{","}\NormalTok{,lambda,}\StringTok{"/m) curve of 50,000 means where m=10,"}\NormalTok{,lambda,}\StringTok{"=1"}\NormalTok{))  ,}\DataTypeTok{xlab=}\StringTok{"sample poisson means"}\NormalTok{ ,}\DataTypeTok{xlim=}\KeywordTok{c}\NormalTok{(}\OperatorTok{-}\FloatTok{0.1}\NormalTok{,}\FloatTok{2.5}\NormalTok{) ,}\DataTypeTok{ylab=}\StringTok{"density"}\NormalTok{ ,}\DataTypeTok{ylim=}\KeywordTok{c}\NormalTok{(}\DecValTok{0}\NormalTok{,}\DecValTok{4}\NormalTok{) ,}\DataTypeTok{freq =} \OtherTok{FALSE}\NormalTok{, }\DataTypeTok{col=}\KeywordTok{viridis}\NormalTok{(}\DecValTok{19}\NormalTok{))}
\NormalTok{x <-}\StringTok{ }\KeywordTok{seq}\NormalTok{(}\DecValTok{0}\NormalTok{,}\DecValTok{2}\NormalTok{, }\DataTypeTok{by =} \FloatTok{0.2}\NormalTok{)}
\NormalTok{y10 <-}\StringTok{ }\KeywordTok{dnorm}\NormalTok{(x, }\DataTypeTok{mean=}\NormalTok{lambda}\FloatTok{.1}\NormalTok{, }\DataTypeTok{sd=}\NormalTok{lambda}\FloatTok{.1}\OperatorTok{/}\NormalTok{(}\KeywordTok{sqrt}\NormalTok{(m10)))}
\KeywordTok{curve}\NormalTok{(}\KeywordTok{dnorm}\NormalTok{(x, }\DataTypeTok{mean=}\NormalTok{lambda}\FloatTok{.1}\NormalTok{, }\DataTypeTok{sd=}\NormalTok{(lambda}\FloatTok{.1}\OperatorTok{/}\KeywordTok{sqrt}\NormalTok{(m10))), }\DataTypeTok{add=}\OtherTok{TRUE}\NormalTok{, }\DataTypeTok{col=}\StringTok{"#cc0000"}\NormalTok{, }\DataTypeTok{lwd=}\DecValTok{2}\NormalTok{)}
\end{Highlighting}
\end{Shaded}

\includegraphics{RProjectST501_PDF_files/figure-latex/unnamed-chunk-16-1.pdf}

\begin{Shaded}
\begin{Highlighting}[]
\NormalTok{TF_values}\FloatTok{.10}\NormalTok{ <-}\StringTok{ }\NormalTok{rMeans10 }\OperatorTok{>=}\StringTok{ }\NormalTok{(lambda}\FloatTok{.1}\OperatorTok{+}\NormalTok{((}\DecValTok{2}\OperatorTok{*}\NormalTok{lambda}\FloatTok{.1}\NormalTok{)}\OperatorTok{/}\KeywordTok{sqrt}\NormalTok{(m10)))}
\NormalTok{TFprob10 <-}\StringTok{ }\KeywordTok{sum}\NormalTok{(TF_values}\FloatTok{.10}\NormalTok{)}\OperatorTok{/}\NormalTok{M; }\KeywordTok{print}\NormalTok{(TFprob10)}
\end{Highlighting}
\end{Shaded}

\begin{verbatim}
## [1] 0.0668
\end{verbatim}

\begin{Shaded}
\begin{Highlighting}[]
\NormalTok{NormProb10 <-}\StringTok{ }\KeywordTok{dnorm}\NormalTok{((lambda}\FloatTok{.1}\OperatorTok{+}\NormalTok{((}\DecValTok{2}\OperatorTok{*}\NormalTok{lambda}\FloatTok{.1}\NormalTok{)}\OperatorTok{/}\KeywordTok{sqrt}\NormalTok{(m10))),}\DataTypeTok{mean=}\NormalTok{lambda}\FloatTok{.1}\NormalTok{,}\DataTypeTok{sd=}\NormalTok{(lambda}\FloatTok{.1}\OperatorTok{/}\KeywordTok{sqrt}\NormalTok{(m10))); }\KeywordTok{print}\NormalTok{(NormProb10)}
\end{Highlighting}
\end{Shaded}

\begin{verbatim}
## [1] 0.1707344
\end{verbatim}

FOR m = 30,\(\lambda\) = 1

\begin{Shaded}
\begin{Highlighting}[]
\NormalTok{m30 =}\StringTok{ }\DecValTok{30}\NormalTok{; lambda}\FloatTok{.1}\NormalTok{ =}\StringTok{ }\DecValTok{1}\NormalTok{; M =}\StringTok{ }\DecValTok{30000}
\NormalTok{ds30 <-}\StringTok{ }\KeywordTok{matrix}\NormalTok{(}\KeywordTok{replicate}\NormalTok{(M,(}\KeywordTok{rpois}\NormalTok{(m30,}\DecValTok{1}\NormalTok{))),}\DataTypeTok{nrow=}\DecValTok{30000}\NormalTok{,}\DataTypeTok{ncol=}\DecValTok{5}\NormalTok{)}
\NormalTok{rMeans30 <-}\StringTok{ }\KeywordTok{matrix}\NormalTok{(}\KeywordTok{rowMeans}\NormalTok{(ds30),}\DataTypeTok{nrow=}\DecValTok{30000}\NormalTok{,}\DataTypeTok{ncol=}\DecValTok{1}\NormalTok{)}
\NormalTok{binlen <-}\StringTok{ }\KeywordTok{c}\NormalTok{(}\OperatorTok{-}\FloatTok{5.1}\NormalTok{,}\OperatorTok{-}\FloatTok{4.9}\NormalTok{,}\OperatorTok{-}\FloatTok{4.7}\NormalTok{,}\OperatorTok{-}\FloatTok{4.5}\NormalTok{,}\OperatorTok{-}\FloatTok{4.3}\NormalTok{,}\OperatorTok{-}\FloatTok{4.1}\NormalTok{,}\OperatorTok{-}\FloatTok{3.9}\NormalTok{,}\OperatorTok{-}\FloatTok{3.7}\NormalTok{,}\OperatorTok{-}\FloatTok{3.5}\NormalTok{,}\OperatorTok{-}\FloatTok{3.3}\NormalTok{,}\OperatorTok{-}\FloatTok{3.1}\NormalTok{,}\OperatorTok{-}\FloatTok{2.9}\NormalTok{,}\OperatorTok{-}\FloatTok{2.7}\NormalTok{,}\OperatorTok{-}\FloatTok{2.5}\NormalTok{,}\OperatorTok{-}\FloatTok{2.3}\NormalTok{,}\OperatorTok{-}\FloatTok{2.1}\NormalTok{,}\OperatorTok{-}\FloatTok{1.9}\NormalTok{,}\OperatorTok{-}\FloatTok{1.7}\NormalTok{,}\OperatorTok{-}\FloatTok{1.5}\NormalTok{,}\OperatorTok{-}\FloatTok{1.3}\NormalTok{,}\OperatorTok{-}\FloatTok{1.1}\NormalTok{,}\OperatorTok{-}\FloatTok{0.9}\NormalTok{,}\OperatorTok{-}\FloatTok{0.7}\NormalTok{,}\OperatorTok{-}\FloatTok{0.5}\NormalTok{,}\OperatorTok{-}\FloatTok{0.3}\NormalTok{,}\OperatorTok{-}\FloatTok{0.1}\NormalTok{,}\FloatTok{0.1}\NormalTok{,}\FloatTok{0.3}\NormalTok{,}\FloatTok{0.5}\NormalTok{,}\FloatTok{0.7}\NormalTok{,}\FloatTok{0.9}\NormalTok{,}\FloatTok{1.1}\NormalTok{,}\FloatTok{1.3}\NormalTok{,}\FloatTok{1.5}\NormalTok{,}\FloatTok{1.7}\NormalTok{,}\FloatTok{1.9}\NormalTok{,}\FloatTok{2.1}\NormalTok{,}\FloatTok{2.3}\NormalTok{,}\FloatTok{2.5}\NormalTok{,}\FloatTok{2.7}\NormalTok{,}\FloatTok{2.9}\NormalTok{,}\FloatTok{3.1}\NormalTok{,}\FloatTok{3.3}\NormalTok{,}\FloatTok{3.5}\NormalTok{,}\FloatTok{3.7}\NormalTok{,}\FloatTok{3.9}\NormalTok{,}\FloatTok{4.1}\NormalTok{,}\FloatTok{4.3}\NormalTok{,}\FloatTok{4.5}\NormalTok{,}\FloatTok{4.7}\NormalTok{,}\FloatTok{4.9}\NormalTok{,}\FloatTok{5.1}\NormalTok{)}
\KeywordTok{hist}\NormalTok{(rMeans30,}\DataTypeTok{breaks=}\NormalTok{binlen ,}\DataTypeTok{main=}\KeywordTok{expression}\NormalTok{(}\KeywordTok{paste}\NormalTok{(}\StringTok{"Pois("}\NormalTok{,lambda,}\StringTok{") ~ N("}\NormalTok{,lambda,}\StringTok{","}\NormalTok{,lambda,}\StringTok{"/m) curve of 50,000 means where m=30,"}\NormalTok{,lambda,}\StringTok{"=1"}\NormalTok{))  ,}\DataTypeTok{xlab=}\StringTok{"sample poisson means"}\NormalTok{ ,}\DataTypeTok{xlim=}\KeywordTok{c}\NormalTok{(}\OperatorTok{-}\FloatTok{0.1}\NormalTok{,}\FloatTok{2.5}\NormalTok{) ,}\DataTypeTok{ylab=}\StringTok{"density"}\NormalTok{ ,}\DataTypeTok{ylim=}\KeywordTok{c}\NormalTok{(}\DecValTok{0}\NormalTok{,}\DecValTok{4}\NormalTok{) ,}\DataTypeTok{freq =} \OtherTok{FALSE}\NormalTok{, }\DataTypeTok{col=}\KeywordTok{viridis}\NormalTok{(}\DecValTok{19}\NormalTok{))}
\NormalTok{x <-}\StringTok{ }\KeywordTok{seq}\NormalTok{(}\DecValTok{0}\NormalTok{,}\DecValTok{2}\NormalTok{, }\DataTypeTok{by =} \FloatTok{0.2}\NormalTok{)}
\NormalTok{y30 <-}\StringTok{ }\KeywordTok{dnorm}\NormalTok{(x, }\DataTypeTok{mean=}\NormalTok{lambda}\FloatTok{.1}\NormalTok{, }\DataTypeTok{sd=}\NormalTok{lambda}\FloatTok{.1}\OperatorTok{/}\NormalTok{(}\KeywordTok{sqrt}\NormalTok{(m30)))}
\KeywordTok{curve}\NormalTok{(}\KeywordTok{dnorm}\NormalTok{(x, }\DataTypeTok{mean=}\NormalTok{lambda}\FloatTok{.1}\NormalTok{, }\DataTypeTok{sd=}\NormalTok{(lambda}\FloatTok{.1}\OperatorTok{/}\KeywordTok{sqrt}\NormalTok{(m30))), }\DataTypeTok{add=}\OtherTok{TRUE}\NormalTok{, }\DataTypeTok{col=}\StringTok{"#FF0033"}\NormalTok{, }\DataTypeTok{lwd=}\DecValTok{2}\NormalTok{)}
\end{Highlighting}
\end{Shaded}

\includegraphics{RProjectST501_PDF_files/figure-latex/unnamed-chunk-17-1.pdf}

\begin{Shaded}
\begin{Highlighting}[]
\NormalTok{TF_values}\FloatTok{.30}\NormalTok{ <-}\StringTok{ }\NormalTok{rMeans30 }\OperatorTok{>=}\StringTok{ }\NormalTok{(lambda}\FloatTok{.1}\OperatorTok{+}\NormalTok{((}\DecValTok{2}\OperatorTok{*}\NormalTok{lambda}\FloatTok{.1}\NormalTok{)}\OperatorTok{/}\KeywordTok{sqrt}\NormalTok{(m30)))}
\NormalTok{TFprob30 <-}\StringTok{ }\KeywordTok{sum}\NormalTok{(TF_values}\FloatTok{.30}\NormalTok{)}\OperatorTok{/}\NormalTok{M; }\KeywordTok{print}\NormalTok{(TFprob30)}
\end{Highlighting}
\end{Shaded}

\begin{verbatim}
## [1] 0.238
\end{verbatim}

\begin{Shaded}
\begin{Highlighting}[]
\NormalTok{NormProb30 <-}\StringTok{ }\KeywordTok{dnorm}\NormalTok{((lambda}\FloatTok{.1}\OperatorTok{+}\NormalTok{((}\DecValTok{2}\OperatorTok{*}\NormalTok{lambda}\FloatTok{.1}\NormalTok{)}\OperatorTok{/}\KeywordTok{sqrt}\NormalTok{(m30))),}\DataTypeTok{mean=}\NormalTok{lambda}\FloatTok{.1}\NormalTok{,}\DataTypeTok{sd=}\NormalTok{(lambda}\FloatTok{.1}\OperatorTok{/}\KeywordTok{sqrt}\NormalTok{(m30))); }\KeywordTok{print}\NormalTok{(NormProb30)}
\end{Highlighting}
\end{Shaded}

\begin{verbatim}
## [1] 0.2957207
\end{verbatim}

FOR m = 100,\(\lambda\) = 1

\begin{Shaded}
\begin{Highlighting}[]
\NormalTok{m100 =}\StringTok{ }\DecValTok{100}\NormalTok{; lambda}\FloatTok{.1}\NormalTok{ =}\StringTok{ }\DecValTok{1}\NormalTok{; M =}\StringTok{ }\DecValTok{50000}
\NormalTok{ds100 <-}\StringTok{ }\KeywordTok{matrix}\NormalTok{(}\KeywordTok{replicate}\NormalTok{(M,(}\KeywordTok{rpois}\NormalTok{(m100,}\DecValTok{1}\NormalTok{))),}\DataTypeTok{nrow=}\DecValTok{50000}\NormalTok{,}\DataTypeTok{ncol=}\DecValTok{5}\NormalTok{)}
\NormalTok{rMeans100 <-}\StringTok{ }\KeywordTok{matrix}\NormalTok{(}\KeywordTok{rowMeans}\NormalTok{(ds100),}\DataTypeTok{nrow=}\DecValTok{50000}\NormalTok{,}\DataTypeTok{ncol=}\DecValTok{1}\NormalTok{)}
\NormalTok{binlen <-}\StringTok{ }\KeywordTok{c}\NormalTok{(}\OperatorTok{-}\FloatTok{5.1}\NormalTok{,}\OperatorTok{-}\FloatTok{4.9}\NormalTok{,}\OperatorTok{-}\FloatTok{4.7}\NormalTok{,}\OperatorTok{-}\FloatTok{4.5}\NormalTok{,}\OperatorTok{-}\FloatTok{4.3}\NormalTok{,}\OperatorTok{-}\FloatTok{4.1}\NormalTok{,}\OperatorTok{-}\FloatTok{3.9}\NormalTok{,}\OperatorTok{-}\FloatTok{3.7}\NormalTok{,}\OperatorTok{-}\FloatTok{3.5}\NormalTok{,}\OperatorTok{-}\FloatTok{3.3}\NormalTok{,}\OperatorTok{-}\FloatTok{3.1}\NormalTok{,}\OperatorTok{-}\FloatTok{2.9}\NormalTok{,}\OperatorTok{-}\FloatTok{2.7}\NormalTok{,}\OperatorTok{-}\FloatTok{2.5}\NormalTok{,}\OperatorTok{-}\FloatTok{2.3}\NormalTok{,}\OperatorTok{-}\FloatTok{2.1}\NormalTok{,}\OperatorTok{-}\FloatTok{1.9}\NormalTok{,}\OperatorTok{-}\FloatTok{1.7}\NormalTok{,}\OperatorTok{-}\FloatTok{1.5}\NormalTok{,}\OperatorTok{-}\FloatTok{1.3}\NormalTok{,}\OperatorTok{-}\FloatTok{1.1}\NormalTok{,}\OperatorTok{-}\FloatTok{0.9}\NormalTok{,}\OperatorTok{-}\FloatTok{0.7}\NormalTok{,}\OperatorTok{-}\FloatTok{0.5}\NormalTok{,}\OperatorTok{-}\FloatTok{0.3}\NormalTok{,}\OperatorTok{-}\FloatTok{0.1}\NormalTok{,}\FloatTok{0.1}\NormalTok{,}\FloatTok{0.3}\NormalTok{,}\FloatTok{0.5}\NormalTok{,}\FloatTok{0.7}\NormalTok{,}\FloatTok{0.9}\NormalTok{,}\FloatTok{1.1}\NormalTok{,}\FloatTok{1.3}\NormalTok{,}\FloatTok{1.5}\NormalTok{,}\FloatTok{1.7}\NormalTok{,}\FloatTok{1.9}\NormalTok{,}\FloatTok{2.1}\NormalTok{,}\FloatTok{2.3}\NormalTok{,}\FloatTok{2.5}\NormalTok{,}\FloatTok{2.7}\NormalTok{,}\FloatTok{2.9}\NormalTok{,}\FloatTok{3.1}\NormalTok{,}\FloatTok{3.3}\NormalTok{,}\FloatTok{3.5}\NormalTok{,}\FloatTok{3.7}\NormalTok{,}\FloatTok{3.9}\NormalTok{,}\FloatTok{4.1}\NormalTok{,}\FloatTok{4.3}\NormalTok{,}\FloatTok{4.5}\NormalTok{,}\FloatTok{4.7}\NormalTok{,}\FloatTok{4.9}\NormalTok{,}\FloatTok{5.1}\NormalTok{)}
\KeywordTok{hist}\NormalTok{(rMeans100,}\DataTypeTok{breaks=}\NormalTok{binlen ,}\DataTypeTok{main=}\KeywordTok{expression}\NormalTok{(}\KeywordTok{paste}\NormalTok{(}\StringTok{"Pois("}\NormalTok{,lambda,}\StringTok{") ~ N("}\NormalTok{,lambda,}\StringTok{","}\NormalTok{,lambda,}\StringTok{"/m) curve of 50,000 means where m=100,"}\NormalTok{,lambda,}\StringTok{"=1"}\NormalTok{))  ,}\DataTypeTok{xlab=}\StringTok{"sample poisson means"}\NormalTok{ ,}\DataTypeTok{xlim=}\KeywordTok{c}\NormalTok{(}\OperatorTok{-}\FloatTok{0.1}\NormalTok{,}\FloatTok{2.5}\NormalTok{) ,}\DataTypeTok{ylab=}\StringTok{"density"}\NormalTok{ ,}\DataTypeTok{ylim=}\KeywordTok{c}\NormalTok{(}\DecValTok{0}\NormalTok{,}\DecValTok{4}\NormalTok{) ,}\DataTypeTok{freq =} \OtherTok{FALSE}\NormalTok{, }\DataTypeTok{col=}\KeywordTok{viridis}\NormalTok{(}\DecValTok{19}\NormalTok{))}
\NormalTok{x <-}\StringTok{ }\KeywordTok{seq}\NormalTok{(}\DecValTok{0}\NormalTok{,}\DecValTok{2}\NormalTok{, }\DataTypeTok{by =} \FloatTok{0.2}\NormalTok{)}
\NormalTok{y100 <-}\StringTok{ }\KeywordTok{dnorm}\NormalTok{(x, }\DataTypeTok{mean=}\NormalTok{lambda}\FloatTok{.1}\NormalTok{, }\DataTypeTok{sd=}\NormalTok{lambda}\FloatTok{.1}\OperatorTok{/}\NormalTok{(}\KeywordTok{sqrt}\NormalTok{(m100)))}
\KeywordTok{curve}\NormalTok{(}\KeywordTok{dnorm}\NormalTok{(x, }\DataTypeTok{mean=}\NormalTok{lambda}\FloatTok{.1}\NormalTok{, }\DataTypeTok{sd=}\NormalTok{(lambda}\FloatTok{.1}\OperatorTok{/}\KeywordTok{sqrt}\NormalTok{(m100))), }\DataTypeTok{add=}\OtherTok{TRUE}\NormalTok{, }\DataTypeTok{col=}\StringTok{"#FF0033"}\NormalTok{, }\DataTypeTok{lwd=}\DecValTok{2}\NormalTok{)}
\end{Highlighting}
\end{Shaded}

\includegraphics{RProjectST501_PDF_files/figure-latex/unnamed-chunk-18-1.pdf}

\begin{Shaded}
\begin{Highlighting}[]
\NormalTok{TF_values}\FloatTok{.100}\NormalTok{ <-}\StringTok{ }\NormalTok{rMeans100 }\OperatorTok{>=}\StringTok{ }\NormalTok{(lambda}\FloatTok{.1}\OperatorTok{+}\NormalTok{((}\DecValTok{2}\OperatorTok{*}\NormalTok{lambda}\FloatTok{.1}\NormalTok{)}\OperatorTok{/}\KeywordTok{sqrt}\NormalTok{(m100)))}
\NormalTok{TFprob100 <-}\StringTok{ }\KeywordTok{sum}\NormalTok{(TF_values}\FloatTok{.100}\NormalTok{)}\OperatorTok{/}\NormalTok{M; }\KeywordTok{print}\NormalTok{(TFprob100)}
\end{Highlighting}
\end{Shaded}

\begin{verbatim}
## [1] 0.38108
\end{verbatim}

\begin{Shaded}
\begin{Highlighting}[]
\NormalTok{NormProb100 <-}\StringTok{ }\KeywordTok{dnorm}\NormalTok{((lambda}\FloatTok{.1}\OperatorTok{+}\NormalTok{((}\DecValTok{2}\OperatorTok{*}\NormalTok{lambda}\FloatTok{.1}\NormalTok{)}\OperatorTok{/}\KeywordTok{sqrt}\NormalTok{(m100))),}\DataTypeTok{mean=}\NormalTok{lambda}\FloatTok{.1}\NormalTok{,}\DataTypeTok{sd=}\NormalTok{(lambda}\FloatTok{.1}\OperatorTok{/}\KeywordTok{sqrt}\NormalTok{(m100))); }\KeywordTok{print}\NormalTok{(NormProb100)}
\end{Highlighting}
\end{Shaded}

\begin{verbatim}
## [1] 0.5399097
\end{verbatim}

Repeat all of the above for \(\lambda\) = 5 and \(\lambda\) = 25. You
should have a total of 12 scenarios/plots FOR m = 1,\(\lambda\) = 5

\begin{Shaded}
\begin{Highlighting}[]
\NormalTok{m1 =}\StringTok{ }\DecValTok{1}\NormalTok{; lambda}\FloatTok{.5}\NormalTok{ =}\StringTok{ }\DecValTok{5}\NormalTok{; M =}\StringTok{ }\DecValTok{50000}
\NormalTok{ds1 <-}\StringTok{ }\KeywordTok{matrix}\NormalTok{(}\KeywordTok{replicate}\NormalTok{(M,(}\KeywordTok{rpois}\NormalTok{(m1,lambda}\FloatTok{.5}\NormalTok{))),}\DataTypeTok{nrow=}\DecValTok{50000}\NormalTok{,}\DataTypeTok{ncol=}\DecValTok{5}\NormalTok{)}
\NormalTok{rMeans1 <-}\StringTok{ }\KeywordTok{matrix}\NormalTok{(}\KeywordTok{rowMeans}\NormalTok{(ds1),}\DataTypeTok{nrow=}\DecValTok{50000}\NormalTok{,}\DataTypeTok{ncol=}\DecValTok{1}\NormalTok{)}
\NormalTok{binlen <-}\StringTok{ }\KeywordTok{c}\NormalTok{(}\OperatorTok{-}\FloatTok{0.5}\NormalTok{,}\FloatTok{0.5}\NormalTok{,}\FloatTok{1.5}\NormalTok{,}\FloatTok{2.5}\NormalTok{,}\FloatTok{3.5}\NormalTok{,}\FloatTok{4.5}\NormalTok{,}\FloatTok{5.5}\NormalTok{,}\FloatTok{6.5}\NormalTok{,}\FloatTok{7.5}\NormalTok{,}\FloatTok{8.5}\NormalTok{,}\FloatTok{9.5}\NormalTok{,}\FloatTok{10.5}\NormalTok{,}\FloatTok{11.5}\NormalTok{,}\FloatTok{12.5}\NormalTok{,}\FloatTok{13.5}\NormalTok{,}\FloatTok{14.5}\NormalTok{,}\FloatTok{15.5}\NormalTok{,}\FloatTok{16.5}\NormalTok{,}\FloatTok{17.5}\NormalTok{,}\FloatTok{18.5}\NormalTok{,}\FloatTok{19.5}\NormalTok{,}\FloatTok{20.5}\NormalTok{,}\FloatTok{21.5}\NormalTok{,}\FloatTok{22.5}\NormalTok{,}\FloatTok{23.5}\NormalTok{,}\FloatTok{24.5}\NormalTok{,}\FloatTok{25.5}\NormalTok{,}\FloatTok{26.5}\NormalTok{,}\FloatTok{27.5}\NormalTok{,}\FloatTok{28.5}\NormalTok{,}\FloatTok{29.5}\NormalTok{,}\FloatTok{30.5}\NormalTok{)}
\KeywordTok{hist}\NormalTok{(rMeans1,}\DataTypeTok{breaks=}\NormalTok{binlen ,}\DataTypeTok{main=}\KeywordTok{expression}\NormalTok{(}\KeywordTok{paste}\NormalTok{(}\StringTok{"Pois("}\NormalTok{,lambda,}\StringTok{") ~ N("}\NormalTok{,lambda,}\StringTok{","}\NormalTok{,lambda,}\StringTok{"/m) curve of 50,000 means where m=1,"}\NormalTok{,lambda,}\StringTok{"=5"}\NormalTok{))  ,}\DataTypeTok{xlab=}\StringTok{"sample poisson means"}\NormalTok{ ,}\DataTypeTok{xlim=}\KeywordTok{c}\NormalTok{(}\DecValTok{0}\NormalTok{,}\DecValTok{16}\NormalTok{) ,}\DataTypeTok{ylab=}\StringTok{"density"}\NormalTok{ ,}\DataTypeTok{ylim=}\KeywordTok{c}\NormalTok{(}\DecValTok{0}\NormalTok{,}\FloatTok{0.3}\NormalTok{) ,}\DataTypeTok{freq =} \OtherTok{FALSE}\NormalTok{, }\DataTypeTok{col=}\KeywordTok{magma}\NormalTok{(}\DecValTok{13}\NormalTok{))}
\NormalTok{x <-}\StringTok{ }\KeywordTok{seq}\NormalTok{(}\OperatorTok{-}\DecValTok{4}\NormalTok{,}\DecValTok{5}\NormalTok{, }\DataTypeTok{by =} \FloatTok{0.1}\NormalTok{)}
\NormalTok{y1 <-}\StringTok{ }\KeywordTok{dnorm}\NormalTok{(x, }\DataTypeTok{mean =}\NormalTok{ lambda}\FloatTok{.5}\NormalTok{, }\DataTypeTok{sd =}\NormalTok{ lambda}\FloatTok{.5}\OperatorTok{/}\KeywordTok{sqrt}\NormalTok{(m1))}
\KeywordTok{curve}\NormalTok{(}\KeywordTok{dnorm}\NormalTok{(x, }\DataTypeTok{mean=}\NormalTok{lambda}\FloatTok{.5}\NormalTok{, }\DataTypeTok{sd=}\NormalTok{lambda}\FloatTok{.5}\OperatorTok{/}\KeywordTok{sqrt}\NormalTok{(m1)), }\DataTypeTok{add=}\OtherTok{TRUE}\NormalTok{, }\DataTypeTok{col=}\StringTok{"#66CC00"}\NormalTok{, }\DataTypeTok{lwd=}\DecValTok{2}\NormalTok{)}
\end{Highlighting}
\end{Shaded}

\includegraphics{RProjectST501_PDF_files/figure-latex/unnamed-chunk-19-1.pdf}

\begin{Shaded}
\begin{Highlighting}[]
\NormalTok{TF_values}\FloatTok{.1}\NormalTok{ <-}\StringTok{ }\NormalTok{rMeans1 }\OperatorTok{>=}\StringTok{ }\NormalTok{(lambda}\FloatTok{.5}\OperatorTok{+}\NormalTok{((}\DecValTok{2}\OperatorTok{*}\NormalTok{lambda}\FloatTok{.5}\NormalTok{)}\OperatorTok{/}\KeywordTok{sqrt}\NormalTok{(m1)))}
\NormalTok{TFprob1 <-}\StringTok{ }\KeywordTok{sum}\NormalTok{(TF_values}\FloatTok{.1}\NormalTok{)}\OperatorTok{/}\NormalTok{M; }\KeywordTok{print}\NormalTok{(TFprob1)}
\end{Highlighting}
\end{Shaded}

\begin{verbatim}
## [1] 0.00016
\end{verbatim}

\begin{Shaded}
\begin{Highlighting}[]
\NormalTok{NormProb1 <-}\StringTok{ }\KeywordTok{dnorm}\NormalTok{((lambda}\FloatTok{.5}\OperatorTok{+}\NormalTok{((}\DecValTok{2}\OperatorTok{*}\NormalTok{lambda}\FloatTok{.5}\NormalTok{)}\OperatorTok{/}\KeywordTok{sqrt}\NormalTok{(m1))),}\DataTypeTok{mean=}\NormalTok{lambda}\FloatTok{.5}\NormalTok{,}\DataTypeTok{sd=}\NormalTok{(lambda}\FloatTok{.5}\OperatorTok{/}\KeywordTok{sqrt}\NormalTok{(m1))); }\KeywordTok{print}\NormalTok{(NormProb1)}
\end{Highlighting}
\end{Shaded}

\begin{verbatim}
## [1] 0.01079819
\end{verbatim}

FOR m = 10,\(\lambda\) = 5

\begin{Shaded}
\begin{Highlighting}[]
\NormalTok{m10 =}\StringTok{ }\DecValTok{10}\NormalTok{; lambda}\FloatTok{.5}\NormalTok{ =}\StringTok{ }\DecValTok{5}\NormalTok{; M =}\StringTok{ }\DecValTok{50000}
\NormalTok{ds10 <-}\StringTok{ }\KeywordTok{matrix}\NormalTok{(}\KeywordTok{replicate}\NormalTok{(M,(}\KeywordTok{rpois}\NormalTok{(m10,lambda}\FloatTok{.5}\NormalTok{))),}\DataTypeTok{nrow=}\DecValTok{50000}\NormalTok{,}\DataTypeTok{ncol=}\DecValTok{5}\NormalTok{)}
\NormalTok{rMeans10 <-}\StringTok{ }\KeywordTok{matrix}\NormalTok{(}\KeywordTok{rowMeans}\NormalTok{(ds10),}\DataTypeTok{nrow=}\DecValTok{50000}\NormalTok{,}\DataTypeTok{ncol=}\DecValTok{1}\NormalTok{)}
\NormalTok{binlen <-}\StringTok{ }\KeywordTok{c}\NormalTok{(}\OperatorTok{-}\FloatTok{0.1}\NormalTok{,}\FloatTok{0.1}\NormalTok{,}\FloatTok{0.3}\NormalTok{,}\FloatTok{0.5}\NormalTok{,}\FloatTok{0.7}\NormalTok{,}\FloatTok{0.9}\NormalTok{,}\FloatTok{1.1}\NormalTok{,}\FloatTok{1.3}\NormalTok{,}\FloatTok{1.5}\NormalTok{,}\FloatTok{1.7}\NormalTok{,}\FloatTok{1.9}\NormalTok{,}\FloatTok{2.1}\NormalTok{,}\FloatTok{2.3}\NormalTok{,}\FloatTok{2.5}\NormalTok{,}\FloatTok{2.7}\NormalTok{,}\FloatTok{2.9}\NormalTok{,}\FloatTok{3.1}\NormalTok{,}\FloatTok{3.3}\NormalTok{,}\FloatTok{3.5}\NormalTok{,}\FloatTok{3.7}\NormalTok{,}\FloatTok{3.9}\NormalTok{,}\FloatTok{4.1}\NormalTok{,}\FloatTok{4.3}\NormalTok{,}\FloatTok{4.5}\NormalTok{,}\FloatTok{4.7}\NormalTok{,}\FloatTok{4.9}\NormalTok{,}\FloatTok{5.1}\NormalTok{,}\FloatTok{5.3}\NormalTok{,}\FloatTok{5.5}\NormalTok{,}\FloatTok{5.7}\NormalTok{,}\FloatTok{5.9}\NormalTok{,}\FloatTok{6.1}\NormalTok{,}\FloatTok{6.3}\NormalTok{,}\FloatTok{6.5}\NormalTok{,}\FloatTok{6.7}\NormalTok{,}\FloatTok{6.9}\NormalTok{,}\FloatTok{7.1}\NormalTok{,}\FloatTok{7.3}\NormalTok{,}\FloatTok{7.5}\NormalTok{,}\FloatTok{7.7}\NormalTok{,}\FloatTok{7.9}\NormalTok{,}\FloatTok{8.1}\NormalTok{,}\FloatTok{8.3}\NormalTok{,}\FloatTok{8.5}\NormalTok{,}\FloatTok{8.7}\NormalTok{,}\FloatTok{8.9}\NormalTok{,}\FloatTok{9.1}\NormalTok{,}\FloatTok{9.3}\NormalTok{,}\FloatTok{9.5}\NormalTok{,}\FloatTok{9.7}\NormalTok{,}\FloatTok{9.9}\NormalTok{,}\FloatTok{10.1}\NormalTok{,}\FloatTok{10.3}\NormalTok{,}\FloatTok{10.5}\NormalTok{,}\FloatTok{10.7}\NormalTok{,}\FloatTok{10.9}\NormalTok{,}\FloatTok{11.1}\NormalTok{,}\FloatTok{11.3}\NormalTok{,}\FloatTok{11.5}\NormalTok{,}\FloatTok{11.7}\NormalTok{,}\FloatTok{11.9}\NormalTok{)}
\KeywordTok{hist}\NormalTok{(rMeans10,}\DataTypeTok{breaks=}\NormalTok{binlen ,}\DataTypeTok{main=}\KeywordTok{expression}\NormalTok{(}\KeywordTok{paste}\NormalTok{(}\StringTok{"Pois("}\NormalTok{,lambda,}\StringTok{") ~ N("}\NormalTok{,lambda,}\StringTok{","}\NormalTok{,lambda,}\StringTok{"/m) curve of 50,000 means where m=10,"}\NormalTok{,lambda,}\StringTok{"=5"}\NormalTok{))  ,}\DataTypeTok{xlab=}\StringTok{"sample poisson means"}\NormalTok{ ,}\DataTypeTok{xlim=}\KeywordTok{c}\NormalTok{(}\DecValTok{2}\NormalTok{,}\FloatTok{8.5}\NormalTok{) ,}\DataTypeTok{ylab=}\StringTok{"density"}\NormalTok{ ,}\DataTypeTok{ylim=}\KeywordTok{c}\NormalTok{(}\DecValTok{0}\NormalTok{,}\FloatTok{1.1}\NormalTok{) ,}\DataTypeTok{freq =} \OtherTok{FALSE}\NormalTok{, }\DataTypeTok{col=}\KeywordTok{magma}\NormalTok{(}\DecValTok{50}\NormalTok{))}
\NormalTok{x <-}\StringTok{ }\KeywordTok{seq}\NormalTok{(}\OperatorTok{-}\DecValTok{4}\NormalTok{,}\DecValTok{5}\NormalTok{, }\DataTypeTok{by =} \FloatTok{0.1}\NormalTok{)}
\NormalTok{y10 <-}\StringTok{ }\KeywordTok{dnorm}\NormalTok{(x, }\DataTypeTok{mean =}\NormalTok{ lambda}\FloatTok{.5}\NormalTok{, }\DataTypeTok{sd =}\NormalTok{ lambda}\FloatTok{.5}\OperatorTok{/}\KeywordTok{sqrt}\NormalTok{(m10))}
\KeywordTok{curve}\NormalTok{(}\KeywordTok{dnorm}\NormalTok{(x, }\DataTypeTok{mean=}\NormalTok{lambda}\FloatTok{.5}\NormalTok{, }\DataTypeTok{sd=}\NormalTok{lambda}\FloatTok{.5}\OperatorTok{/}\KeywordTok{sqrt}\NormalTok{(m10)), }\DataTypeTok{add=}\OtherTok{TRUE}\NormalTok{, }\DataTypeTok{col=}\StringTok{"#66CC00"}\NormalTok{, }\DataTypeTok{lwd=}\DecValTok{2}\NormalTok{)}
\end{Highlighting}
\end{Shaded}

\includegraphics{RProjectST501_PDF_files/figure-latex/unnamed-chunk-20-1.pdf}

\begin{Shaded}
\begin{Highlighting}[]
\NormalTok{TF_values}\FloatTok{.10}\NormalTok{ <-}\StringTok{ }\NormalTok{rMeans10 }\OperatorTok{>=}\StringTok{ }\NormalTok{(lambda}\FloatTok{.5}\OperatorTok{+}\NormalTok{((}\DecValTok{2}\OperatorTok{*}\NormalTok{lambda}\FloatTok{.5}\NormalTok{)}\OperatorTok{/}\KeywordTok{sqrt}\NormalTok{(m10)))}
\NormalTok{TFprob10 <-}\StringTok{ }\KeywordTok{sum}\NormalTok{(TF_values}\FloatTok{.10}\NormalTok{)}\OperatorTok{/}\NormalTok{M; }\KeywordTok{print}\NormalTok{(TFprob10)}
\end{Highlighting}
\end{Shaded}

\begin{verbatim}
## [1] 0.0018
\end{verbatim}

\begin{Shaded}
\begin{Highlighting}[]
\NormalTok{NormProb10 <-}\StringTok{ }\KeywordTok{dnorm}\NormalTok{((lambda}\FloatTok{.5}\OperatorTok{+}\NormalTok{((}\DecValTok{2}\OperatorTok{*}\NormalTok{lambda}\FloatTok{.5}\NormalTok{)}\OperatorTok{/}\KeywordTok{sqrt}\NormalTok{(m10))),}\DataTypeTok{mean=}\NormalTok{lambda}\FloatTok{.5}\NormalTok{,}\DataTypeTok{sd=}\NormalTok{(lambda}\FloatTok{.5}\OperatorTok{/}\KeywordTok{sqrt}\NormalTok{(m10))); }\KeywordTok{print}\NormalTok{(NormProb10)}
\end{Highlighting}
\end{Shaded}

\begin{verbatim}
## [1] 0.03414689
\end{verbatim}

FOR m = 30,\(\lambda\) = 5

\begin{Shaded}
\begin{Highlighting}[]
\NormalTok{m30 =}\StringTok{ }\DecValTok{30}\NormalTok{; lambda}\FloatTok{.5}\NormalTok{ =}\StringTok{ }\DecValTok{5}\NormalTok{; M =}\StringTok{ }\DecValTok{30000}
\NormalTok{ds30 <-}\StringTok{ }\KeywordTok{matrix}\NormalTok{(}\KeywordTok{replicate}\NormalTok{(M,(}\KeywordTok{rpois}\NormalTok{(m30,lambda}\FloatTok{.5}\NormalTok{))),}\DataTypeTok{nrow=}\DecValTok{30000}\NormalTok{,}\DataTypeTok{ncol=}\DecValTok{5}\NormalTok{)}
\NormalTok{rMeans30 <-}\StringTok{ }\KeywordTok{matrix}\NormalTok{(}\KeywordTok{rowMeans}\NormalTok{(ds30),}\DataTypeTok{nrow=}\DecValTok{30000}\NormalTok{,}\DataTypeTok{ncol=}\DecValTok{1}\NormalTok{)}
\NormalTok{binlen <-}\StringTok{ }\KeywordTok{c}\NormalTok{(}\OperatorTok{-}\FloatTok{0.1}\NormalTok{,}\FloatTok{0.1}\NormalTok{,}\FloatTok{0.3}\NormalTok{,}\FloatTok{0.5}\NormalTok{,}\FloatTok{0.7}\NormalTok{,}\FloatTok{0.9}\NormalTok{,}\FloatTok{1.1}\NormalTok{,}\FloatTok{1.3}\NormalTok{,}\FloatTok{1.5}\NormalTok{,}\FloatTok{1.7}\NormalTok{,}\FloatTok{1.9}\NormalTok{,}\FloatTok{2.1}\NormalTok{,}\FloatTok{2.3}\NormalTok{,}\FloatTok{2.5}\NormalTok{,}\FloatTok{2.7}\NormalTok{,}\FloatTok{2.9}\NormalTok{,}\FloatTok{3.1}\NormalTok{,}\FloatTok{3.3}\NormalTok{,}\FloatTok{3.5}\NormalTok{,}\FloatTok{3.7}\NormalTok{,}\FloatTok{3.9}\NormalTok{,}\FloatTok{4.1}\NormalTok{,}\FloatTok{4.3}\NormalTok{,}\FloatTok{4.5}\NormalTok{,}\FloatTok{4.7}\NormalTok{,}\FloatTok{4.9}\NormalTok{,}\FloatTok{5.1}\NormalTok{,}\FloatTok{5.3}\NormalTok{,}\FloatTok{5.5}\NormalTok{,}\FloatTok{5.7}\NormalTok{,}\FloatTok{5.9}\NormalTok{,}\FloatTok{6.1}\NormalTok{,}\FloatTok{6.3}\NormalTok{,}\FloatTok{6.5}\NormalTok{,}\FloatTok{6.7}\NormalTok{,}\FloatTok{6.9}\NormalTok{,}\FloatTok{7.1}\NormalTok{,}\FloatTok{7.3}\NormalTok{,}\FloatTok{7.5}\NormalTok{,}\FloatTok{7.7}\NormalTok{,}\FloatTok{7.9}\NormalTok{,}\FloatTok{8.1}\NormalTok{,}\FloatTok{8.3}\NormalTok{,}\FloatTok{8.5}\NormalTok{,}\FloatTok{8.7}\NormalTok{,}\FloatTok{8.9}\NormalTok{,}\FloatTok{9.1}\NormalTok{,}\FloatTok{9.3}\NormalTok{,}\FloatTok{9.5}\NormalTok{,}\FloatTok{9.7}\NormalTok{,}\FloatTok{9.9}\NormalTok{,}\FloatTok{10.1}\NormalTok{,}\FloatTok{10.3}\NormalTok{,}\FloatTok{10.5}\NormalTok{,}\FloatTok{10.7}\NormalTok{,}\FloatTok{10.9}\NormalTok{,}\FloatTok{11.1}\NormalTok{,}\FloatTok{11.3}\NormalTok{,}\FloatTok{11.5}\NormalTok{,}\FloatTok{11.7}\NormalTok{,}\FloatTok{11.9}\NormalTok{)}
\KeywordTok{hist}\NormalTok{(rMeans30,}\DataTypeTok{breaks=}\NormalTok{binlen ,}\DataTypeTok{main=}\KeywordTok{expression}\NormalTok{(}\KeywordTok{paste}\NormalTok{(}\StringTok{"Pois("}\NormalTok{,lambda,}\StringTok{") ~ N("}\NormalTok{,lambda,}\StringTok{","}\NormalTok{,lambda,}\StringTok{"/m) curve of 50,000 means where m=30,"}\NormalTok{,lambda,}\StringTok{"=5"}\NormalTok{))  ,}\DataTypeTok{xlab=}\StringTok{"sample poisson means"}\NormalTok{ ,}\DataTypeTok{xlim=}\KeywordTok{c}\NormalTok{(}\DecValTok{2}\NormalTok{,}\FloatTok{8.5}\NormalTok{) ,}\DataTypeTok{ylab=}\StringTok{"density"}\NormalTok{ ,}\DataTypeTok{ylim=}\KeywordTok{c}\NormalTok{(}\DecValTok{0}\NormalTok{,}\FloatTok{1.1}\NormalTok{) ,}\DataTypeTok{freq =} \OtherTok{FALSE}\NormalTok{, }\DataTypeTok{col=}\KeywordTok{magma}\NormalTok{(}\DecValTok{50}\NormalTok{))}
\NormalTok{x30 <-}\StringTok{ }\KeywordTok{seq}\NormalTok{(}\OperatorTok{-}\DecValTok{4}\NormalTok{,}\DecValTok{5}\NormalTok{, }\DataTypeTok{by =} \FloatTok{0.1}\NormalTok{)}
\NormalTok{y30 <-}\StringTok{ }\KeywordTok{dnorm}\NormalTok{(x30, }\DataTypeTok{mean =}\NormalTok{ lambda}\FloatTok{.5}\NormalTok{, }\DataTypeTok{sd =}\NormalTok{ lambda}\FloatTok{.5}\OperatorTok{/}\KeywordTok{sqrt}\NormalTok{(m30))}
\KeywordTok{curve}\NormalTok{(}\KeywordTok{dnorm}\NormalTok{(x, }\DataTypeTok{mean=}\NormalTok{lambda}\FloatTok{.5}\NormalTok{, }\DataTypeTok{sd=}\NormalTok{lambda}\FloatTok{.5}\OperatorTok{/}\KeywordTok{sqrt}\NormalTok{(m30)), }\DataTypeTok{add=}\OtherTok{TRUE}\NormalTok{, }\DataTypeTok{col=}\StringTok{"#66CC00"}\NormalTok{, }\DataTypeTok{lwd=}\DecValTok{2}\NormalTok{)}
\end{Highlighting}
\end{Shaded}

\includegraphics{RProjectST501_PDF_files/figure-latex/unnamed-chunk-21-1.pdf}

\begin{Shaded}
\begin{Highlighting}[]
\NormalTok{TF_values}\FloatTok{.30}\NormalTok{ <-}\StringTok{ }\NormalTok{rMeans30 }\OperatorTok{>=}\StringTok{ }\NormalTok{(lambda}\FloatTok{.5}\OperatorTok{+}\NormalTok{((}\DecValTok{2}\OperatorTok{*}\NormalTok{lambda}\FloatTok{.5}\NormalTok{)}\OperatorTok{/}\KeywordTok{sqrt}\NormalTok{(m30)))}
\NormalTok{TFprob30 <-}\StringTok{ }\KeywordTok{sum}\NormalTok{(TF_values}\FloatTok{.30}\NormalTok{)}\OperatorTok{/}\NormalTok{M;}\KeywordTok{print}\NormalTok{(TFprob30)}
\end{Highlighting}
\end{Shaded}

\begin{verbatim}
## [1] 0.03346667
\end{verbatim}

\begin{Shaded}
\begin{Highlighting}[]
\NormalTok{NormProb30 <-}\StringTok{ }\KeywordTok{dnorm}\NormalTok{((lambda}\FloatTok{.5}\OperatorTok{+}\NormalTok{((}\DecValTok{2}\OperatorTok{*}\NormalTok{lambda}\FloatTok{.5}\NormalTok{)}\OperatorTok{/}\KeywordTok{sqrt}\NormalTok{(m30))),}\DataTypeTok{mean=}\NormalTok{lambda}\FloatTok{.5}\NormalTok{,}\DataTypeTok{sd=}\NormalTok{(lambda}\FloatTok{.5}\OperatorTok{/}\KeywordTok{sqrt}\NormalTok{(m30)));}\KeywordTok{print}\NormalTok{(NormProb30)}
\end{Highlighting}
\end{Shaded}

\begin{verbatim}
## [1] 0.05914414
\end{verbatim}

FOR m = 100,\(\lambda\) = 5

\begin{Shaded}
\begin{Highlighting}[]
\NormalTok{m100 =}\StringTok{ }\DecValTok{100}\NormalTok{; lambda}\FloatTok{.5}\NormalTok{ =}\StringTok{ }\DecValTok{5}\NormalTok{; M =}\StringTok{ }\DecValTok{50000}
\NormalTok{ds100 <-}\StringTok{ }\KeywordTok{matrix}\NormalTok{(}\KeywordTok{replicate}\NormalTok{(M,(}\KeywordTok{rpois}\NormalTok{(m100,lambda}\FloatTok{.5}\NormalTok{))),}\DataTypeTok{nrow=}\DecValTok{50000}\NormalTok{,}\DataTypeTok{ncol=}\DecValTok{5}\NormalTok{)}
\NormalTok{rMeans100 <-}\StringTok{ }\KeywordTok{matrix}\NormalTok{(}\KeywordTok{rowMeans}\NormalTok{(ds100),}\DataTypeTok{nrow=}\DecValTok{50000}\NormalTok{,}\DataTypeTok{ncol=}\DecValTok{1}\NormalTok{)}
\NormalTok{binlen <-}\StringTok{ }\KeywordTok{c}\NormalTok{(}\OperatorTok{-}\FloatTok{0.1}\NormalTok{,}\FloatTok{0.1}\NormalTok{,}\FloatTok{0.3}\NormalTok{,}\FloatTok{0.5}\NormalTok{,}\FloatTok{0.7}\NormalTok{,}\FloatTok{0.9}\NormalTok{,}\FloatTok{1.1}\NormalTok{,}\FloatTok{1.3}\NormalTok{,}\FloatTok{1.5}\NormalTok{,}\FloatTok{1.7}\NormalTok{,}\FloatTok{1.9}\NormalTok{,}\FloatTok{2.1}\NormalTok{,}\FloatTok{2.3}\NormalTok{,}\FloatTok{2.5}\NormalTok{,}\FloatTok{2.7}\NormalTok{,}\FloatTok{2.9}\NormalTok{,}\FloatTok{3.1}\NormalTok{,}\FloatTok{3.3}\NormalTok{,}\FloatTok{3.5}\NormalTok{,}\FloatTok{3.7}\NormalTok{,}\FloatTok{3.9}\NormalTok{,}\FloatTok{4.1}\NormalTok{,}\FloatTok{4.3}\NormalTok{,}\FloatTok{4.5}\NormalTok{,}\FloatTok{4.7}\NormalTok{,}\FloatTok{4.9}\NormalTok{,}\FloatTok{5.1}\NormalTok{,}\FloatTok{5.3}\NormalTok{,}\FloatTok{5.5}\NormalTok{,}\FloatTok{5.7}\NormalTok{,}\FloatTok{5.9}\NormalTok{,}\FloatTok{6.1}\NormalTok{,}\FloatTok{6.3}\NormalTok{,}\FloatTok{6.5}\NormalTok{,}\FloatTok{6.7}\NormalTok{,}\FloatTok{6.9}\NormalTok{,}\FloatTok{7.1}\NormalTok{,}\FloatTok{7.3}\NormalTok{,}\FloatTok{7.5}\NormalTok{,}\FloatTok{7.7}\NormalTok{,}\FloatTok{7.9}\NormalTok{,}\FloatTok{8.1}\NormalTok{,}\FloatTok{8.3}\NormalTok{,}\FloatTok{8.5}\NormalTok{,}\FloatTok{8.7}\NormalTok{,}\FloatTok{8.9}\NormalTok{,}\FloatTok{9.1}\NormalTok{,}\FloatTok{9.3}\NormalTok{,}\FloatTok{9.5}\NormalTok{,}\FloatTok{9.7}\NormalTok{,}\FloatTok{9.9}\NormalTok{,}\FloatTok{10.1}\NormalTok{,}\FloatTok{10.3}\NormalTok{,}\FloatTok{10.5}\NormalTok{,}\FloatTok{10.7}\NormalTok{,}\FloatTok{10.9}\NormalTok{,}\FloatTok{11.1}\NormalTok{,}\FloatTok{11.3}\NormalTok{,}\FloatTok{11.5}\NormalTok{,}\FloatTok{11.7}\NormalTok{,}\FloatTok{11.9}\NormalTok{)}
\KeywordTok{hist}\NormalTok{(rMeans100,}\DataTypeTok{breaks=}\NormalTok{binlen ,}\DataTypeTok{main=}\KeywordTok{expression}\NormalTok{(}\KeywordTok{paste}\NormalTok{(}\StringTok{"Pois("}\NormalTok{,lambda,}\StringTok{") ~ N("}\NormalTok{,lambda,}\StringTok{","}\NormalTok{,lambda,}\StringTok{"/m) curve of 50,000 means where m=100,"}\NormalTok{,lambda,}\StringTok{"=5"}\NormalTok{))  ,}\DataTypeTok{xlab=}\StringTok{"sample poisson means"}\NormalTok{ ,}\DataTypeTok{xlim=}\KeywordTok{c}\NormalTok{(}\DecValTok{2}\NormalTok{,}\FloatTok{8.5}\NormalTok{) ,}\DataTypeTok{ylab=}\StringTok{"density"}\NormalTok{ ,}\DataTypeTok{ylim=}\KeywordTok{c}\NormalTok{(}\DecValTok{0}\NormalTok{,}\FloatTok{1.1}\NormalTok{) ,}\DataTypeTok{freq =} \OtherTok{FALSE}\NormalTok{, }\DataTypeTok{col=}\KeywordTok{magma}\NormalTok{(}\DecValTok{50}\NormalTok{))}
\NormalTok{x100 <-}\StringTok{ }\KeywordTok{seq}\NormalTok{(}\OperatorTok{-}\DecValTok{4}\NormalTok{,}\DecValTok{5}\NormalTok{, }\DataTypeTok{by =} \FloatTok{0.1}\NormalTok{)}
\NormalTok{y100 <-}\StringTok{ }\KeywordTok{dnorm}\NormalTok{(x100, }\DataTypeTok{mean =}\NormalTok{ lambda}\FloatTok{.5}\NormalTok{, }\DataTypeTok{sd =}\NormalTok{ lambda}\FloatTok{.5}\OperatorTok{/}\KeywordTok{sqrt}\NormalTok{(m100))}
\KeywordTok{curve}\NormalTok{(}\KeywordTok{dnorm}\NormalTok{(x, }\DataTypeTok{mean=}\NormalTok{lambda}\FloatTok{.5}\NormalTok{, }\DataTypeTok{sd=}\NormalTok{lambda}\FloatTok{.5}\OperatorTok{/}\KeywordTok{sqrt}\NormalTok{(m100)), }\DataTypeTok{add=}\OtherTok{TRUE}\NormalTok{, }\DataTypeTok{col=}\StringTok{"#66CC00"}\NormalTok{, }\DataTypeTok{lwd=}\DecValTok{2}\NormalTok{)}
\end{Highlighting}
\end{Shaded}

\includegraphics{RProjectST501_PDF_files/figure-latex/unnamed-chunk-22-1.pdf}

\begin{Shaded}
\begin{Highlighting}[]
\NormalTok{TF_values}\FloatTok{.100}\NormalTok{ <-}\StringTok{ }\NormalTok{rMeans100 }\OperatorTok{>=}\StringTok{ }\NormalTok{(lambda}\FloatTok{.5}\OperatorTok{+}\NormalTok{((}\DecValTok{2}\OperatorTok{*}\NormalTok{lambda}\FloatTok{.5}\NormalTok{)}\OperatorTok{/}\KeywordTok{sqrt}\NormalTok{(m100)))}
\NormalTok{TFprob100 <-}\StringTok{ }\KeywordTok{sum}\NormalTok{(TF_values}\FloatTok{.100}\NormalTok{)}\OperatorTok{/}\NormalTok{M;}\KeywordTok{print}\NormalTok{(TFprob100)}
\end{Highlighting}
\end{Shaded}

\begin{verbatim}
## [1] 0.18026
\end{verbatim}

\begin{Shaded}
\begin{Highlighting}[]
\NormalTok{NormProb100 <-}\StringTok{ }\KeywordTok{dnorm}\NormalTok{((lambda}\FloatTok{.5}\OperatorTok{+}\NormalTok{((}\DecValTok{2}\OperatorTok{*}\NormalTok{lambda}\FloatTok{.5}\NormalTok{)}\OperatorTok{/}\KeywordTok{sqrt}\NormalTok{(m100))),}\DataTypeTok{mean=}\NormalTok{lambda}\FloatTok{.5}\NormalTok{,}\DataTypeTok{sd=}\NormalTok{(lambda}\FloatTok{.5}\OperatorTok{/}\KeywordTok{sqrt}\NormalTok{(m100)));}\KeywordTok{print}\NormalTok{(NormProb100)}
\end{Highlighting}
\end{Shaded}

\begin{verbatim}
## [1] 0.1079819
\end{verbatim}

FOR m = 1,\(\lambda\) = 25

\begin{Shaded}
\begin{Highlighting}[]
\NormalTok{m1 =}\StringTok{ }\DecValTok{1}\NormalTok{; lambda}\FloatTok{.25}\NormalTok{ =}\StringTok{ }\DecValTok{25}\NormalTok{; M =}\StringTok{ }\DecValTok{50000}
\NormalTok{ds1 <-}\StringTok{ }\KeywordTok{matrix}\NormalTok{(}\KeywordTok{replicate}\NormalTok{(M,(}\KeywordTok{rpois}\NormalTok{(m1,lambda}\FloatTok{.25}\NormalTok{))),}\DataTypeTok{nrow=}\DecValTok{50000}\NormalTok{,}\DataTypeTok{ncol=}\DecValTok{5}\NormalTok{)}
\NormalTok{rMeans1 <-}\StringTok{ }\KeywordTok{matrix}\NormalTok{(}\KeywordTok{rowMeans}\NormalTok{(ds1),}\DataTypeTok{nrow=}\DecValTok{50000}\NormalTok{,}\DataTypeTok{ncol=}\DecValTok{1}\NormalTok{)}
\KeywordTok{hist}\NormalTok{(rMeans1, }\DataTypeTok{breaks=}\DecValTok{40}\NormalTok{,}\DataTypeTok{main=}\KeywordTok{expression}\NormalTok{(}\KeywordTok{paste}\NormalTok{(}\StringTok{"Pois("}\NormalTok{,lambda,}\StringTok{") ~ N("}\NormalTok{,lambda,}\StringTok{","}\NormalTok{,lambda,}\StringTok{"/m) curve of 50,000 means where m=1,"}\NormalTok{,lambda,}\StringTok{"=25"}\NormalTok{))  ,}\DataTypeTok{xlab=}\StringTok{"sample poisson means"}\NormalTok{ ,}\DataTypeTok{xlim=}\KeywordTok{c}\NormalTok{(}\DecValTok{7}\NormalTok{,}\DecValTok{47}\NormalTok{) ,}\DataTypeTok{ylab=}\StringTok{"density"}\NormalTok{ ,}\DataTypeTok{ylim=}\KeywordTok{c}\NormalTok{(}\DecValTok{0}\NormalTok{,}\FloatTok{0.2}\NormalTok{) ,}\DataTypeTok{freq =} \OtherTok{FALSE}\NormalTok{, }\DataTypeTok{col=}\KeywordTok{heat.colors}\NormalTok{(}\DecValTok{40}\NormalTok{))}
\NormalTok{x <-}\StringTok{ }\KeywordTok{seq}\NormalTok{(}\OperatorTok{-}\DecValTok{4}\NormalTok{,}\DecValTok{5}\NormalTok{, }\DataTypeTok{by =} \FloatTok{0.1}\NormalTok{)}
\NormalTok{y1 <-}\StringTok{ }\KeywordTok{dnorm}\NormalTok{(x, }\DataTypeTok{mean =}\NormalTok{ lambda}\FloatTok{.25}\NormalTok{, }\DataTypeTok{sd =}\NormalTok{ lambda}\FloatTok{.25}\OperatorTok{/}\KeywordTok{sqrt}\NormalTok{(m1))}
\KeywordTok{curve}\NormalTok{(}\KeywordTok{dnorm}\NormalTok{(x, }\DataTypeTok{mean=}\NormalTok{lambda}\FloatTok{.25}\NormalTok{, }\DataTypeTok{sd=}\NormalTok{lambda}\FloatTok{.25}\OperatorTok{/}\KeywordTok{sqrt}\NormalTok{(m1)), }\DataTypeTok{add=}\OtherTok{TRUE}\NormalTok{, }\DataTypeTok{col=}\StringTok{"blue"}\NormalTok{, }\DataTypeTok{lwd=}\DecValTok{2}\NormalTok{)}
\end{Highlighting}
\end{Shaded}

\includegraphics{RProjectST501_PDF_files/figure-latex/unnamed-chunk-23-1.pdf}

\begin{Shaded}
\begin{Highlighting}[]
\NormalTok{TF_values}\FloatTok{.1}\NormalTok{ <-}\StringTok{ }\NormalTok{rMeans1 }\OperatorTok{>=}\StringTok{ }\NormalTok{(lambda}\FloatTok{.25}\OperatorTok{+}\NormalTok{((}\DecValTok{2}\OperatorTok{*}\NormalTok{lambda}\FloatTok{.25}\NormalTok{)}\OperatorTok{/}\KeywordTok{sqrt}\NormalTok{(m1)))}
\NormalTok{TFprob1 <-}\StringTok{ }\KeywordTok{sum}\NormalTok{(TF_values}\FloatTok{.1}\NormalTok{)}\OperatorTok{/}\NormalTok{M; }\KeywordTok{print}\NormalTok{(TFprob1)}
\end{Highlighting}
\end{Shaded}

\begin{verbatim}
## [1] 0
\end{verbatim}

\begin{Shaded}
\begin{Highlighting}[]
\NormalTok{NormProb1 <-}\StringTok{ }\KeywordTok{dnorm}\NormalTok{((lambda}\FloatTok{.25}\OperatorTok{+}\NormalTok{((}\DecValTok{2}\OperatorTok{*}\NormalTok{lambda}\FloatTok{.25}\NormalTok{)}\OperatorTok{/}\KeywordTok{sqrt}\NormalTok{(m1))),}\DataTypeTok{mean=}\NormalTok{lambda}\FloatTok{.25}\NormalTok{,}\DataTypeTok{sd=}\NormalTok{(lambda}\FloatTok{.25}\OperatorTok{/}\KeywordTok{sqrt}\NormalTok{(m1))); }\KeywordTok{print}\NormalTok{(NormProb1)}
\end{Highlighting}
\end{Shaded}

\begin{verbatim}
## [1] 0.002159639
\end{verbatim}

FOR m = 10,\(\lambda\) = 25

\begin{Shaded}
\begin{Highlighting}[]
\NormalTok{m10 =}\StringTok{ }\DecValTok{10}\NormalTok{; lambda}\FloatTok{.25}\NormalTok{ =}\StringTok{ }\DecValTok{25}\NormalTok{; M =}\StringTok{ }\DecValTok{50000}
\NormalTok{ds10 <-}\StringTok{ }\KeywordTok{matrix}\NormalTok{(}\KeywordTok{replicate}\NormalTok{(M,(}\KeywordTok{rpois}\NormalTok{(m10,lambda}\FloatTok{.25}\NormalTok{))),}\DataTypeTok{nrow=}\DecValTok{50000}\NormalTok{,}\DataTypeTok{ncol=}\DecValTok{5}\NormalTok{)}
\NormalTok{rMeans10 <-}\StringTok{ }\KeywordTok{matrix}\NormalTok{(}\KeywordTok{rowMeans}\NormalTok{(ds10),}\DataTypeTok{nrow=}\DecValTok{50000}\NormalTok{,}\DataTypeTok{ncol=}\DecValTok{1}\NormalTok{)}
\NormalTok{binlen <-}\StringTok{ }\KeywordTok{c}\NormalTok{(}\FloatTok{15.1}\NormalTok{,}\FloatTok{15.3}\NormalTok{,}\FloatTok{15.5}\NormalTok{,}\FloatTok{15.7}\NormalTok{,}\FloatTok{15.9}\NormalTok{,}\FloatTok{16.1}\NormalTok{,}\FloatTok{16.3}\NormalTok{,}\FloatTok{16.5}\NormalTok{,}\FloatTok{16.7}\NormalTok{,}\FloatTok{16.9}\NormalTok{,}\FloatTok{17.1}\NormalTok{,}\FloatTok{17.3}\NormalTok{,}\FloatTok{17.5}\NormalTok{,}\FloatTok{17.7}\NormalTok{,}\FloatTok{17.9}\NormalTok{,}\FloatTok{18.1}\NormalTok{,}\FloatTok{18.3}\NormalTok{,}\FloatTok{18.5}\NormalTok{,}\FloatTok{18.7}\NormalTok{,}\FloatTok{18.9}\NormalTok{,}\FloatTok{19.1}\NormalTok{,}\FloatTok{19.3}\NormalTok{,}\FloatTok{19.5}\NormalTok{,}\FloatTok{19.7}\NormalTok{,}\FloatTok{19.9}\NormalTok{,}\FloatTok{20.1}\NormalTok{,}\FloatTok{20.3}\NormalTok{,}\FloatTok{20.5}\NormalTok{,}\FloatTok{20.7}\NormalTok{,}\FloatTok{20.9}\NormalTok{,}\FloatTok{21.1}\NormalTok{,}\FloatTok{21.3}\NormalTok{,}\FloatTok{21.5}\NormalTok{,}\FloatTok{21.7}\NormalTok{,}\FloatTok{21.9}\NormalTok{,}\FloatTok{22.1}\NormalTok{,}\FloatTok{22.3}\NormalTok{,}\FloatTok{22.5}\NormalTok{,}\FloatTok{22.7}\NormalTok{,}\FloatTok{22.9}\NormalTok{,}\FloatTok{23.1}\NormalTok{,}\FloatTok{23.3}\NormalTok{,}\FloatTok{23.5}\NormalTok{,}\FloatTok{23.7}\NormalTok{,}\FloatTok{23.9}\NormalTok{,}\FloatTok{24.1}\NormalTok{,}\FloatTok{24.3}\NormalTok{,}\FloatTok{24.5}\NormalTok{,}\FloatTok{24.7}\NormalTok{,}\FloatTok{24.9}\NormalTok{,}\FloatTok{25.1}\NormalTok{,}\FloatTok{25.3}\NormalTok{,}\FloatTok{25.5}\NormalTok{,}\FloatTok{25.7}\NormalTok{,}\FloatTok{25.9}\NormalTok{,}\FloatTok{26.1}\NormalTok{,}\FloatTok{26.3}\NormalTok{,}\FloatTok{26.5}\NormalTok{,}\FloatTok{26.7}\NormalTok{,}\FloatTok{26.9}\NormalTok{,}\FloatTok{27.1}\NormalTok{,}\FloatTok{27.3}\NormalTok{,}\FloatTok{27.5}\NormalTok{,}\FloatTok{27.7}\NormalTok{,}\FloatTok{27.9}\NormalTok{,}\FloatTok{28.1}\NormalTok{,}\FloatTok{28.3}\NormalTok{,}\FloatTok{28.5}\NormalTok{,}\FloatTok{28.7}\NormalTok{,}\FloatTok{28.9}\NormalTok{,}\FloatTok{29.1}\NormalTok{,}\FloatTok{29.3}\NormalTok{,}\FloatTok{29.5}\NormalTok{,}\FloatTok{29.7}\NormalTok{,}\FloatTok{29.9}\NormalTok{,}\FloatTok{30.1}\NormalTok{,}\FloatTok{30.3}\NormalTok{,}\FloatTok{30.5}\NormalTok{,}\FloatTok{30.7}\NormalTok{,}\FloatTok{30.9}\NormalTok{,}\FloatTok{31.1}\NormalTok{,}\FloatTok{31.3}\NormalTok{,}\FloatTok{31.5}\NormalTok{,}\FloatTok{31.7}\NormalTok{,}\FloatTok{31.9}\NormalTok{,}\FloatTok{32.1}\NormalTok{,}\FloatTok{32.3}\NormalTok{,}\FloatTok{32.5}\NormalTok{,}\FloatTok{32.7}\NormalTok{,}\FloatTok{32.9}\NormalTok{,}\FloatTok{33.1}\NormalTok{,}\FloatTok{33.3}\NormalTok{,}\FloatTok{33.5}\NormalTok{,}\FloatTok{33.7}\NormalTok{,}\FloatTok{33.9}\NormalTok{,}\FloatTok{34.1}\NormalTok{,}\FloatTok{34.3}\NormalTok{,}\FloatTok{34.5}\NormalTok{,}\FloatTok{34.7}\NormalTok{,}\FloatTok{34.9}\NormalTok{,}\FloatTok{35.1}\NormalTok{,}\FloatTok{35.3}\NormalTok{,}\FloatTok{35.5}\NormalTok{,}\FloatTok{35.7}\NormalTok{,}\FloatTok{35.9}\NormalTok{,}\FloatTok{36.1}\NormalTok{,}\FloatTok{36.3}\NormalTok{,}\FloatTok{36.5}\NormalTok{,}\FloatTok{36.7}\NormalTok{,}\FloatTok{36.9}\NormalTok{,}\FloatTok{37.1}\NormalTok{,}\FloatTok{37.3}\NormalTok{)}
\KeywordTok{hist}\NormalTok{(rMeans10,}\DataTypeTok{breaks=}\NormalTok{binlen ,}\DataTypeTok{main=}\KeywordTok{expression}\NormalTok{(}\KeywordTok{paste}\NormalTok{(}\StringTok{"Pois("}\NormalTok{,lambda,}\StringTok{") ~ N("}\NormalTok{,lambda,}\StringTok{","}\NormalTok{,lambda,}\StringTok{"/m) curve of 50,000 means where m=10,"}\NormalTok{,lambda,}\StringTok{"=25"}\NormalTok{))  ,}\DataTypeTok{xlab=}\StringTok{"sample poisson means"}\NormalTok{ ,}\DataTypeTok{xlim=}\KeywordTok{c}\NormalTok{(}\DecValTok{15}\NormalTok{,}\DecValTok{36}\NormalTok{) ,}\DataTypeTok{ylab=}\StringTok{"density"}\NormalTok{ ,}\DataTypeTok{ylim=}\KeywordTok{c}\NormalTok{(}\DecValTok{0}\NormalTok{,}\FloatTok{0.2}\NormalTok{) ,}\DataTypeTok{freq =} \OtherTok{FALSE}\NormalTok{, }\DataTypeTok{col=}\KeywordTok{heat.colors}\NormalTok{(}\DecValTok{100}\NormalTok{))}
\NormalTok{x <-}\StringTok{ }\KeywordTok{seq}\NormalTok{(}\OperatorTok{-}\DecValTok{4}\NormalTok{,}\DecValTok{5}\NormalTok{, }\DataTypeTok{by =} \FloatTok{0.1}\NormalTok{)}
\NormalTok{y10 <-}\StringTok{ }\KeywordTok{dnorm}\NormalTok{(x, }\DataTypeTok{mean =}\NormalTok{ lambda}\FloatTok{.25}\NormalTok{, }\DataTypeTok{sd =}\NormalTok{ lambda}\FloatTok{.25}\OperatorTok{/}\KeywordTok{sqrt}\NormalTok{(m10))}
\KeywordTok{curve}\NormalTok{(}\KeywordTok{dnorm}\NormalTok{(x, }\DataTypeTok{mean=}\NormalTok{lambda}\FloatTok{.25}\NormalTok{, }\DataTypeTok{sd=}\NormalTok{lambda}\FloatTok{.25}\OperatorTok{/}\KeywordTok{sqrt}\NormalTok{(m10)), }\DataTypeTok{add=}\OtherTok{TRUE}\NormalTok{, }\DataTypeTok{col=}\StringTok{"blue"}\NormalTok{, }\DataTypeTok{lwd=}\DecValTok{2}\NormalTok{)}
\end{Highlighting}
\end{Shaded}

\includegraphics{RProjectST501_PDF_files/figure-latex/unnamed-chunk-24-1.pdf}

\begin{Shaded}
\begin{Highlighting}[]
\NormalTok{TF_values}\FloatTok{.10}\NormalTok{ <-}\StringTok{ }\NormalTok{rMeans10 }\OperatorTok{>=}\StringTok{ }\NormalTok{(lambda}\FloatTok{.25}\OperatorTok{+}\NormalTok{((}\DecValTok{2}\OperatorTok{*}\NormalTok{lambda}\FloatTok{.25}\NormalTok{)}\OperatorTok{/}\KeywordTok{sqrt}\NormalTok{(m10)))}
\NormalTok{TFprob10 <-}\StringTok{ }\KeywordTok{sum}\NormalTok{(TF_values}\FloatTok{.10}\NormalTok{)}\OperatorTok{/}\NormalTok{M; }\KeywordTok{print}\NormalTok{(TFprob10)}
\end{Highlighting}
\end{Shaded}

\begin{verbatim}
## [1] 0
\end{verbatim}

\begin{Shaded}
\begin{Highlighting}[]
\NormalTok{NormProb10 <-}\StringTok{ }\KeywordTok{dnorm}\NormalTok{((lambda}\FloatTok{.25}\OperatorTok{+}\NormalTok{((}\DecValTok{2}\OperatorTok{*}\NormalTok{lambda}\FloatTok{.25}\NormalTok{)}\OperatorTok{/}\KeywordTok{sqrt}\NormalTok{(m10))),}\DataTypeTok{mean=}\NormalTok{lambda}\FloatTok{.25}\NormalTok{,}\DataTypeTok{sd=}\NormalTok{(lambda}\FloatTok{.25}\OperatorTok{/}\KeywordTok{sqrt}\NormalTok{(m10))); }\KeywordTok{print}\NormalTok{(NormProb10)}
\end{Highlighting}
\end{Shaded}

\begin{verbatim}
## [1] 0.006829377
\end{verbatim}

FOR m = 30,\(\lambda\) = 25

\begin{Shaded}
\begin{Highlighting}[]
\NormalTok{m30 =}\StringTok{ }\DecValTok{30}\NormalTok{; lambda}\FloatTok{.25}\NormalTok{ =}\StringTok{ }\DecValTok{25}\NormalTok{; M =}\StringTok{ }\DecValTok{30000}
\NormalTok{ds30 <-}\StringTok{ }\KeywordTok{matrix}\NormalTok{(}\KeywordTok{replicate}\NormalTok{(M,(}\KeywordTok{rpois}\NormalTok{(m30,lambda}\FloatTok{.25}\NormalTok{))),}\DataTypeTok{nrow=}\DecValTok{30000}\NormalTok{,}\DataTypeTok{ncol=}\DecValTok{5}\NormalTok{)}
\NormalTok{rMeans30 <-}\StringTok{ }\KeywordTok{matrix}\NormalTok{(}\KeywordTok{rowMeans}\NormalTok{(ds30),}\DataTypeTok{nrow=}\DecValTok{30000}\NormalTok{,}\DataTypeTok{ncol=}\DecValTok{1}\NormalTok{)}
\NormalTok{binlen <-}\StringTok{ }\KeywordTok{c}\NormalTok{(}\FloatTok{15.1}\NormalTok{,}\FloatTok{15.3}\NormalTok{,}\FloatTok{15.5}\NormalTok{,}\FloatTok{15.7}\NormalTok{,}\FloatTok{15.9}\NormalTok{,}\FloatTok{16.1}\NormalTok{,}\FloatTok{16.3}\NormalTok{,}\FloatTok{16.5}\NormalTok{,}\FloatTok{16.7}\NormalTok{,}\FloatTok{16.9}\NormalTok{,}\FloatTok{17.1}\NormalTok{,}\FloatTok{17.3}\NormalTok{,}\FloatTok{17.5}\NormalTok{,}\FloatTok{17.7}\NormalTok{,}\FloatTok{17.9}\NormalTok{,}\FloatTok{18.1}\NormalTok{,}\FloatTok{18.3}\NormalTok{,}\FloatTok{18.5}\NormalTok{,}\FloatTok{18.7}\NormalTok{,}\FloatTok{18.9}\NormalTok{,}\FloatTok{19.1}\NormalTok{,}\FloatTok{19.3}\NormalTok{,}\FloatTok{19.5}\NormalTok{,}\FloatTok{19.7}\NormalTok{,}\FloatTok{19.9}\NormalTok{,}\FloatTok{20.1}\NormalTok{,}\FloatTok{20.3}\NormalTok{,}\FloatTok{20.5}\NormalTok{,}\FloatTok{20.7}\NormalTok{,}\FloatTok{20.9}\NormalTok{,}\FloatTok{21.1}\NormalTok{,}\FloatTok{21.3}\NormalTok{,}\FloatTok{21.5}\NormalTok{,}\FloatTok{21.7}\NormalTok{,}\FloatTok{21.9}\NormalTok{,}\FloatTok{22.1}\NormalTok{,}\FloatTok{22.3}\NormalTok{,}\FloatTok{22.5}\NormalTok{,}\FloatTok{22.7}\NormalTok{,}\FloatTok{22.9}\NormalTok{,}\FloatTok{23.1}\NormalTok{,}\FloatTok{23.3}\NormalTok{,}\FloatTok{23.5}\NormalTok{,}\FloatTok{23.7}\NormalTok{,}\FloatTok{23.9}\NormalTok{,}\FloatTok{24.1}\NormalTok{,}\FloatTok{24.3}\NormalTok{,}\FloatTok{24.5}\NormalTok{,}\FloatTok{24.7}\NormalTok{,}\FloatTok{24.9}\NormalTok{,}\FloatTok{25.1}\NormalTok{,}\FloatTok{25.3}\NormalTok{,}\FloatTok{25.5}\NormalTok{,}\FloatTok{25.7}\NormalTok{,}\FloatTok{25.9}\NormalTok{,}\FloatTok{26.1}\NormalTok{,}\FloatTok{26.3}\NormalTok{,}\FloatTok{26.5}\NormalTok{,}\FloatTok{26.7}\NormalTok{,}\FloatTok{26.9}\NormalTok{,}\FloatTok{27.1}\NormalTok{,}\FloatTok{27.3}\NormalTok{,}\FloatTok{27.5}\NormalTok{,}\FloatTok{27.7}\NormalTok{,}\FloatTok{27.9}\NormalTok{,}\FloatTok{28.1}\NormalTok{,}\FloatTok{28.3}\NormalTok{,}\FloatTok{28.5}\NormalTok{,}\FloatTok{28.7}\NormalTok{,}\FloatTok{28.9}\NormalTok{,}\FloatTok{29.1}\NormalTok{,}\FloatTok{29.3}\NormalTok{,}\FloatTok{29.5}\NormalTok{,}\FloatTok{29.7}\NormalTok{,}\FloatTok{29.9}\NormalTok{,}\FloatTok{30.1}\NormalTok{,}\FloatTok{30.3}\NormalTok{,}\FloatTok{30.5}\NormalTok{,}\FloatTok{30.7}\NormalTok{,}\FloatTok{30.9}\NormalTok{,}\FloatTok{31.1}\NormalTok{,}\FloatTok{31.3}\NormalTok{,}\FloatTok{31.5}\NormalTok{,}\FloatTok{31.7}\NormalTok{,}\FloatTok{31.9}\NormalTok{,}\FloatTok{32.1}\NormalTok{,}\FloatTok{32.3}\NormalTok{,}\FloatTok{32.5}\NormalTok{,}\FloatTok{32.7}\NormalTok{,}\FloatTok{32.9}\NormalTok{,}\FloatTok{33.1}\NormalTok{,}\FloatTok{33.3}\NormalTok{,}\FloatTok{33.5}\NormalTok{,}\FloatTok{33.7}\NormalTok{,}\FloatTok{33.9}\NormalTok{,}\FloatTok{34.1}\NormalTok{,}\FloatTok{34.3}\NormalTok{,}\FloatTok{34.5}\NormalTok{,}\FloatTok{34.7}\NormalTok{,}\FloatTok{34.9}\NormalTok{,}\FloatTok{35.1}\NormalTok{,}\FloatTok{35.3}\NormalTok{,}\FloatTok{35.5}\NormalTok{,}\FloatTok{35.7}\NormalTok{,}\FloatTok{35.9}\NormalTok{,}\FloatTok{36.1}\NormalTok{,}\FloatTok{36.3}\NormalTok{,}\FloatTok{36.5}\NormalTok{,}\FloatTok{36.7}\NormalTok{,}\FloatTok{36.9}\NormalTok{,}\FloatTok{37.1}\NormalTok{,}\FloatTok{37.3}\NormalTok{)}
\KeywordTok{hist}\NormalTok{(rMeans30,}\DataTypeTok{breaks=}\NormalTok{binlen ,}\DataTypeTok{main=}\KeywordTok{expression}\NormalTok{(}\KeywordTok{paste}\NormalTok{(}\StringTok{"Pois("}\NormalTok{,lambda,}\StringTok{") ~ N("}\NormalTok{,lambda,}\StringTok{","}\NormalTok{,lambda,}\StringTok{"/m) curve of 50,000 means where m=30,"}\NormalTok{,lambda,}\StringTok{"=25"}\NormalTok{))  ,}\DataTypeTok{xlab=}\StringTok{"sample poisson means"}\NormalTok{ ,}\DataTypeTok{xlim=}\KeywordTok{c}\NormalTok{(}\DecValTok{15}\NormalTok{,}\DecValTok{36}\NormalTok{) }
\NormalTok{     ,}\DataTypeTok{ylab=}\StringTok{"density"}\NormalTok{ ,}\DataTypeTok{ylim=}\KeywordTok{c}\NormalTok{(}\DecValTok{0}\NormalTok{,}\FloatTok{0.2}\NormalTok{) ,}\DataTypeTok{freq =} \OtherTok{FALSE}\NormalTok{, }\DataTypeTok{col=}\KeywordTok{heat.colors}\NormalTok{(}\DecValTok{100}\NormalTok{))}
\NormalTok{x30 <-}\StringTok{ }\KeywordTok{seq}\NormalTok{(}\OperatorTok{-}\DecValTok{4}\NormalTok{,}\DecValTok{5}\NormalTok{, }\DataTypeTok{by =} \FloatTok{0.1}\NormalTok{)}
\NormalTok{y30 <-}\StringTok{ }\KeywordTok{dnorm}\NormalTok{(x30, }\DataTypeTok{mean =}\NormalTok{ lambda}\FloatTok{.25}\NormalTok{, }\DataTypeTok{sd =}\NormalTok{ lambda}\FloatTok{.25}\OperatorTok{/}\KeywordTok{sqrt}\NormalTok{(m30))}
\KeywordTok{curve}\NormalTok{(}\KeywordTok{dnorm}\NormalTok{(x, }\DataTypeTok{mean=}\NormalTok{lambda}\FloatTok{.25}\NormalTok{, }\DataTypeTok{sd=}\NormalTok{lambda}\FloatTok{.25}\OperatorTok{/}\KeywordTok{sqrt}\NormalTok{(m30)), }\DataTypeTok{add=}\OtherTok{TRUE}\NormalTok{, }\DataTypeTok{col=}\StringTok{"blue"}\NormalTok{, }\DataTypeTok{lwd=}\DecValTok{2}\NormalTok{)}
\end{Highlighting}
\end{Shaded}

\includegraphics{RProjectST501_PDF_files/figure-latex/unnamed-chunk-25-1.pdf}

\begin{Shaded}
\begin{Highlighting}[]
\NormalTok{TF_values}\FloatTok{.30}\NormalTok{ <-}\StringTok{ }\NormalTok{rMeans30 }\OperatorTok{>=}\StringTok{ }\NormalTok{(lambda}\FloatTok{.25}\OperatorTok{+}\NormalTok{((}\DecValTok{2}\OperatorTok{*}\NormalTok{lambda}\FloatTok{.25}\NormalTok{)}\OperatorTok{/}\KeywordTok{sqrt}\NormalTok{(m30)))}
\NormalTok{TFprob30 <-}\StringTok{ }\KeywordTok{sum}\NormalTok{(TF_values}\FloatTok{.30}\NormalTok{)}\OperatorTok{/}\NormalTok{M; }\KeywordTok{print}\NormalTok{(TFprob30)}
\end{Highlighting}
\end{Shaded}

\begin{verbatim}
## [1] 6.666667e-05
\end{verbatim}

\begin{Shaded}
\begin{Highlighting}[]
\NormalTok{NormProb30 <-}\StringTok{ }\KeywordTok{dnorm}\NormalTok{((lambda}\FloatTok{.25}\OperatorTok{+}\NormalTok{((}\DecValTok{2}\OperatorTok{*}\NormalTok{lambda}\FloatTok{.25}\NormalTok{)}\OperatorTok{/}\KeywordTok{sqrt}\NormalTok{(m30))),}\DataTypeTok{mean=}\NormalTok{lambda}\FloatTok{.25}\NormalTok{,}\DataTypeTok{sd=}\NormalTok{(lambda}\FloatTok{.25}\OperatorTok{/}\KeywordTok{sqrt}\NormalTok{(m30)));}\KeywordTok{print}\NormalTok{(NormProb30)}
\end{Highlighting}
\end{Shaded}

\begin{verbatim}
## [1] 0.01182883
\end{verbatim}

FOR m = 100,\(\lambda\) = 25

\begin{Shaded}
\begin{Highlighting}[]
\NormalTok{m100 =}\StringTok{ }\DecValTok{100}\NormalTok{; lambda}\FloatTok{.25}\NormalTok{ =}\StringTok{ }\DecValTok{25}\NormalTok{; M =}\StringTok{ }\DecValTok{50000}
\NormalTok{ds100 <-}\StringTok{ }\KeywordTok{matrix}\NormalTok{(}\KeywordTok{replicate}\NormalTok{(M,(}\KeywordTok{rpois}\NormalTok{(m100,lambda}\FloatTok{.25}\NormalTok{))),}\DataTypeTok{nrow=}\DecValTok{50000}\NormalTok{,}\DataTypeTok{ncol=}\DecValTok{5}\NormalTok{)}
\NormalTok{rMeans100 <-}\StringTok{ }\KeywordTok{matrix}\NormalTok{(}\KeywordTok{rowMeans}\NormalTok{(ds100),}\DataTypeTok{nrow=}\DecValTok{50000}\NormalTok{,}\DataTypeTok{ncol=}\DecValTok{1}\NormalTok{)}
\NormalTok{binlen <-}\StringTok{ }\KeywordTok{c}\NormalTok{(}\FloatTok{15.1}\NormalTok{,}\FloatTok{15.3}\NormalTok{,}\FloatTok{15.5}\NormalTok{,}\FloatTok{15.7}\NormalTok{,}\FloatTok{15.9}\NormalTok{,}\FloatTok{16.1}\NormalTok{,}\FloatTok{16.3}\NormalTok{,}\FloatTok{16.5}\NormalTok{,}\FloatTok{16.7}\NormalTok{,}\FloatTok{16.9}\NormalTok{,}\FloatTok{17.1}\NormalTok{,}\FloatTok{17.3}\NormalTok{,}\FloatTok{17.5}\NormalTok{,}\FloatTok{17.7}\NormalTok{,}\FloatTok{17.9}\NormalTok{,}\FloatTok{18.1}\NormalTok{,}\FloatTok{18.3}\NormalTok{,}\FloatTok{18.5}\NormalTok{,}\FloatTok{18.7}\NormalTok{,}\FloatTok{18.9}\NormalTok{,}\FloatTok{19.1}\NormalTok{,}\FloatTok{19.3}\NormalTok{,}\FloatTok{19.5}\NormalTok{,}\FloatTok{19.7}\NormalTok{,}\FloatTok{19.9}\NormalTok{,}\FloatTok{20.1}\NormalTok{,}\FloatTok{20.3}\NormalTok{,}\FloatTok{20.5}\NormalTok{,}\FloatTok{20.7}\NormalTok{,}\FloatTok{20.9}\NormalTok{,}\FloatTok{21.1}\NormalTok{,}\FloatTok{21.3}\NormalTok{,}\FloatTok{21.5}\NormalTok{,}\FloatTok{21.7}\NormalTok{,}\FloatTok{21.9}\NormalTok{,}\FloatTok{22.1}\NormalTok{,}\FloatTok{22.3}\NormalTok{,}\FloatTok{22.5}\NormalTok{,}\FloatTok{22.7}\NormalTok{,}\FloatTok{22.9}\NormalTok{,}\FloatTok{23.1}\NormalTok{,}\FloatTok{23.3}\NormalTok{,}\FloatTok{23.5}\NormalTok{,}\FloatTok{23.7}\NormalTok{,}\FloatTok{23.9}\NormalTok{,}\FloatTok{24.1}\NormalTok{,}\FloatTok{24.3}\NormalTok{,}\FloatTok{24.5}\NormalTok{,}\FloatTok{24.7}\NormalTok{,}\FloatTok{24.9}\NormalTok{,}\FloatTok{25.1}\NormalTok{,}\FloatTok{25.3}\NormalTok{,}\FloatTok{25.5}\NormalTok{,}\FloatTok{25.7}\NormalTok{,}\FloatTok{25.9}\NormalTok{,}\FloatTok{26.1}\NormalTok{,}\FloatTok{26.3}\NormalTok{,}\FloatTok{26.5}\NormalTok{,}\FloatTok{26.7}\NormalTok{,}\FloatTok{26.9}\NormalTok{,}\FloatTok{27.1}\NormalTok{,}\FloatTok{27.3}\NormalTok{,}\FloatTok{27.5}\NormalTok{,}\FloatTok{27.7}\NormalTok{,}\FloatTok{27.9}\NormalTok{,}\FloatTok{28.1}\NormalTok{,}\FloatTok{28.3}\NormalTok{,}\FloatTok{28.5}\NormalTok{,}\FloatTok{28.7}\NormalTok{,}\FloatTok{28.9}\NormalTok{,}\FloatTok{29.1}\NormalTok{,}\FloatTok{29.3}\NormalTok{,}\FloatTok{29.5}\NormalTok{,}\FloatTok{29.7}\NormalTok{,}\FloatTok{29.9}\NormalTok{,}\FloatTok{30.1}\NormalTok{,}\FloatTok{30.3}\NormalTok{,}\FloatTok{30.5}\NormalTok{,}\FloatTok{30.7}\NormalTok{,}\FloatTok{30.9}\NormalTok{,}\FloatTok{31.1}\NormalTok{,}\FloatTok{31.3}\NormalTok{,}\FloatTok{31.5}\NormalTok{,}\FloatTok{31.7}\NormalTok{,}\FloatTok{31.9}\NormalTok{,}\FloatTok{32.1}\NormalTok{,}\FloatTok{32.3}\NormalTok{,}\FloatTok{32.5}\NormalTok{,}\FloatTok{32.7}\NormalTok{,}\FloatTok{32.9}\NormalTok{,}\FloatTok{33.1}\NormalTok{,}\FloatTok{33.3}\NormalTok{,}\FloatTok{33.5}\NormalTok{,}\FloatTok{33.7}\NormalTok{,}\FloatTok{33.9}\NormalTok{,}\FloatTok{34.1}\NormalTok{,}\FloatTok{34.3}\NormalTok{,}\FloatTok{34.5}\NormalTok{,}\FloatTok{34.7}\NormalTok{,}\FloatTok{34.9}\NormalTok{,}\FloatTok{35.1}\NormalTok{,}\FloatTok{35.3}\NormalTok{,}\FloatTok{35.5}\NormalTok{,}\FloatTok{35.7}\NormalTok{,}\FloatTok{35.9}\NormalTok{,}\FloatTok{36.1}\NormalTok{,}\FloatTok{36.3}\NormalTok{,}\FloatTok{36.5}\NormalTok{,}\FloatTok{36.7}\NormalTok{,}\FloatTok{36.9}\NormalTok{,}\FloatTok{37.1}\NormalTok{,}\FloatTok{37.3}\NormalTok{)}
\KeywordTok{hist}\NormalTok{(rMeans100,}\DataTypeTok{breaks=}\NormalTok{binlen ,}\DataTypeTok{main=}\KeywordTok{expression}\NormalTok{(}\KeywordTok{paste}\NormalTok{(}\StringTok{"Pois("}\NormalTok{,lambda,}\StringTok{") ~ N("}\NormalTok{,lambda,}\StringTok{","}\NormalTok{,lambda,}\StringTok{"/m) curve of 50,000 means where m=100,"}\NormalTok{,lambda,}\StringTok{"=25"}\NormalTok{))  ,}\DataTypeTok{xlab=}\StringTok{"sample poisson means"}\NormalTok{ ,}\DataTypeTok{xlim=}\KeywordTok{c}\NormalTok{(}\DecValTok{15}\NormalTok{,}\DecValTok{36}\NormalTok{) ,}\DataTypeTok{ylab=}\StringTok{"density"}\NormalTok{ ,}\DataTypeTok{ylim=}\KeywordTok{c}\NormalTok{(}\DecValTok{0}\NormalTok{,}\FloatTok{0.2}\NormalTok{) ,}\DataTypeTok{freq =} \OtherTok{FALSE}\NormalTok{, }\DataTypeTok{col=}\KeywordTok{heat.colors}\NormalTok{(}\DecValTok{100}\NormalTok{))}
\NormalTok{x100 <-}\StringTok{ }\KeywordTok{seq}\NormalTok{(}\OperatorTok{-}\DecValTok{4}\NormalTok{,}\DecValTok{5}\NormalTok{, }\DataTypeTok{by =} \FloatTok{0.1}\NormalTok{)}
\NormalTok{y100 <-}\StringTok{ }\KeywordTok{dnorm}\NormalTok{(x100, }\DataTypeTok{mean =}\NormalTok{ lambda}\FloatTok{.25}\NormalTok{, }\DataTypeTok{sd =}\NormalTok{ lambda}\FloatTok{.25}\OperatorTok{/}\KeywordTok{sqrt}\NormalTok{(m100))}
\KeywordTok{curve}\NormalTok{(}\KeywordTok{dnorm}\NormalTok{(x, }\DataTypeTok{mean=}\NormalTok{lambda}\FloatTok{.25}\NormalTok{, }\DataTypeTok{sd=}\NormalTok{lambda}\FloatTok{.25}\OperatorTok{/}\KeywordTok{sqrt}\NormalTok{(m100)), }\DataTypeTok{add=}\OtherTok{TRUE}\NormalTok{, }\DataTypeTok{col=}\StringTok{"blue"}\NormalTok{, }\DataTypeTok{lwd=}\DecValTok{2}\NormalTok{)}
\end{Highlighting}
\end{Shaded}

\includegraphics{RProjectST501_PDF_files/figure-latex/unnamed-chunk-26-1.pdf}

\begin{Shaded}
\begin{Highlighting}[]
\NormalTok{TF_values}\FloatTok{.100}\NormalTok{ <-}\StringTok{ }\NormalTok{rMeans100 }\OperatorTok{>=}\StringTok{ }\NormalTok{(lambda}\FloatTok{.25}\OperatorTok{+}\NormalTok{((}\DecValTok{2}\OperatorTok{*}\NormalTok{lambda}\FloatTok{.25}\NormalTok{)}\OperatorTok{/}\KeywordTok{sqrt}\NormalTok{(m100)))}
\NormalTok{TFprob100 <-}\StringTok{ }\KeywordTok{sum}\NormalTok{(TF_values}\FloatTok{.100}\NormalTok{)}\OperatorTok{/}\NormalTok{M; }\KeywordTok{print}\NormalTok{(TFprob100)}
\end{Highlighting}
\end{Shaded}

\begin{verbatim}
## [1] 0.01628
\end{verbatim}

\begin{Shaded}
\begin{Highlighting}[]
\NormalTok{NormProb100 <-}\StringTok{ }\KeywordTok{dnorm}\NormalTok{((lambda}\FloatTok{.25}\OperatorTok{+}\NormalTok{((}\DecValTok{2}\OperatorTok{*}\NormalTok{lambda}\FloatTok{.25}\NormalTok{)}\OperatorTok{/}\KeywordTok{sqrt}\NormalTok{(m100))),}\DataTypeTok{mean=}\NormalTok{lambda}\FloatTok{.25}\NormalTok{,}\DataTypeTok{sd=}\NormalTok{(lambda}\FloatTok{.25}\OperatorTok{/}\KeywordTok{sqrt}\NormalTok{(m100)));}\KeywordTok{print}\NormalTok{(NormProb100)}
\end{Highlighting}
\end{Shaded}

\begin{verbatim}
## [1] 0.02159639
\end{verbatim}

\textbf{Discuss how these plots and probabilities can help someone
understand convergence in distribution.}

"It is interesting to see a discrete poisson distribution converging
toward a continuous normal on a single graph. While poisson(\(\lambda\))
is observed at its smallest evaluated values, there are significantly
less individually distinct values a mean can take on in the same
interval length of the largest observed poisson(\(\lambda\)) value. For
example, at \(\lambda\) = 1 from (0,3), a mean can be observed at the
integers 0,1,2,3. At \(\lambda\) = 100, observed mean values on that
same range include rational numbers whose means have intervals of 0.2;
i.e.~0.0,0.2,0.4,0.6,\ldots etc. We can interpret that as having less
variability at \(\lambda\) = 1 as opposed to \(\lambda\) = 100. Even at
lower \(\lambda\) values, we can observe a relatively decent fit of the
normality curve with small sample sizes. The normality curve fits
\(\lambda\) = 1 best betwen m=10 and m=30. It appears that as m=100, the
\(\lambda\) = 1 model show more variability than the curve. The inverse
is usually true when \(\lambda\) is higher and m is lower.

When poisson(\(\lambda\)) reaches \(\lambda\) = 5, we have a beautifully
shaped histogram that looks far more closer to a normally-shaped curve
than \(\lambda\) = 1. At \(\lambda\) = 5, the sample size of 1 is too
small to fit this model well. The curve is too horizontal and
illustrates more variability than the poisson(\(\lambda\)). We will need
more sample. It is not until m=30 that we get our closest curve yet to
the histogram.

Finally, poisson(\(\lambda\))obtains our highest observed \(\lambda\) at
\(\lambda\) = 25 and the m has to be big in order for the
poisson(\(\lambda\)) to converge toward the
normal(\(\lambda\),\(\lambda\)/m). The line is too horizontal at m=1 and
m=10. At m=30 the curve begins to takes shape and by m=100, the curve is
bending just beneath the histogram.

This is an interesting result because of the implications for finding
various statistical parameters for a poisson distribution from a normal
distribution when that poisson distributionconverges to a normal."

\textbf{Why do you think the large-sample approximation works better for
larger \(\lambda\) values?}

With smaller-sample approximation there are fewer observations and thus
more oppurtunity for random values to appear distant from the constant
to which the pois(\(\lambda\)) is expected to converge to. In a small
sample, These values will carry more weight as it pertains to
variability. A large enough sample and a large enough centering mass of
observations about the constant will have a sort of diluting or drowning
out effect on the values further from the converging constant. In turn,
these larger samplings will fit the pois(\(\lambda\)) more closely.

\end{document}
